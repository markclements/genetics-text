\chapter{Genetic Control of Development}\label{genetic-control-of-development}

\href{https://en.wikipedia.org/wiki/Embryonic_development}{Embryonic development} also embryogenesis is the process by which the embryo forms and develops. In mammals, the term refers chiefly to early stages of prenatal development, whereas the terms fetus and fetal development describe later stages.

Embryonic development starts with the fertilization of the egg cell (ovum) by a sperm cell, (spermatozoon). Once fertilized, the ovum is referred to as a zygote, a single diploid cell. The zygote undergoes mitotic divisions with no significant growth (a process known as cleavage) and cellular differentiation, leading to development of a multicellular embryo.

\href{https://en.wikipedia.org/wiki/Evolutionary_developmental_biology}{Evolutionary developmental biology} (informally, evo-devo) is a field of biological research that compares the developmental processes of different organisms to infer the ancestral relationships between them and how developmental processes evolved.

The field grew from 19th-century beginnings, where embryology faced a mystery: zoologists did not know how embryonic development was controlled at the molecular level. Charles Darwin noted that having similar embryos implied common ancestry, but little progress was made until the 1970s. Then, recombinant DNA technology at last brought embryology together with molecular genetics. A key early discovery was of homeotic genes that regulate development in a wide range of eukaryotes.

The field is characterised by some key concepts which took evolutionary biologists by surprise. One is deep homology, the finding that dissimilar organs such as the eyes of insects, vertebrates and cephalopod molluscs, long thought to have evolved separately, are controlled by similar genes such as pax-6, from the evo-devo gene toolkit. These genes are ancient, being highly conserved among phyla; they generate the patterns in time and space which shape the embryo, and ultimately form the body plan of the organism. Another is that species do not differ much in their structural genes, such as those coding for enzymes; what does differ is the way that gene expression is regulated by the toolkit genes. These genes are reused, unchanged, many times in different parts of the embryo and at different stages of development, forming a complex cascade of control, switching other regulatory genes as well as structural genes on and off in a precise pattern. This multiple pleiotropic reuse explains why these genes are highly conserved, as any change would have many adverse consequences which natural selection would oppose.

New morphological features and ultimately new species are produced by variations in the toolkit, either when genes are expressed in a new pattern, or when toolkit genes acquire additional functions. Another possibility is the Neo-Lamarckian theory that epigenetic changes are later consolidated at gene level, something that may have been important early in the history of multicellular life.

\hypertarget{the-control-of-body-structure}{%
\section{The control of body structure}\label{the-control-of-body-structure}}

\hypertarget{deep-homology}{%
\subsection{Deep homology}\label{deep-homology}}

Roughly spherical eggs of different animals give rise to extremely different bodies, from jellyfish to lobsters, butterflies to elephants. Many of these organisms share the same structural genes for body-building proteins like collagen and enzymes, but biologists had expected that each group of animals would have its own rules of development. The surprise of evo-devo is that the shaping of bodies is controlled by a rather small percentage of genes, and that these regulatory genes are ancient, shared by all animals. The giraffe does not have a gene for a long neck, any more than the elephant has a gene for a big body. Their bodies are patterned by a system of switching which causes development of different features to begin earlier or later, to occur in this or that part of the embryo, and to continue for more or less time.

The puzzle of how embryonic development was controlled began to be solved using the fruit fly Drosophila melanogaster as a model organism. The step-by-step control of its embryogenesis was visualized by attaching fluorescent dyes of different colours to specific types of protein made by genes expressed in the embryo. A dye such as green fluorescent protein, originally from a jellyfish, was typically attached to an antibody specific to a fruit fly protein, forming a precise indicator of where and when that protein appeared in the living embryo.

Using such a technique, in 1994 Walter Gehring found that the pax-6 gene, vital for forming the eyes of fruit flies, exactly matches an eye-forming gene in mice and humans. The same gene was quickly found in many other groups of animals, such as squid, a cephalopod mollusc. Biologists including Ernst Mayr had believed that eyes had arisen in the animal kingdom at least 40 times, as the anatomy of different types of eye varies widely. For example, the fruit fly's compound eye is made of hundreds of small lensed structures (ommatidia); the human eye has a blind spot where the optic nerve enters the eye, and the nerve fibres run over the surface of the retina,so light has to pass through a layer of nerve fibres before reaching the detector cells in the retina, so the structure is effectively ``upside-down''; in contrast, the cephalopod eye has the retina, then a layer of nerve fibres, then the wall of the eye ``the right way around''. The evidence of pax-6, however, was that the same genes controlled the development of the eyes of all these animals, suggesting that they all evolved from a common ancestor. Ancient genes had been conserved through millions of years of evolution to create dissimilar structures for similar functions, demonstrating deep homology between structures once thought to be purely analogous. This has caused a radical revision of the meaning of homology in evolutionary biology.

\hypertarget{gene-toolkit}{%
\section{Gene toolkit}\label{gene-toolkit}}

A small fraction of the genes in an organism's genome control the organism's development. These genes are called the developmental-genetic toolkit. They are highly conserved among phyla, meaning that they are ancient and very similar in widely separated groups of animals. Differences in deployment of toolkit genes affect the body plan and the number, identity, and pattern of body parts. Most toolkit genes are parts of signalling pathways: they encode transcription factors, cell adhesion proteins, cell surface receptor proteins and signalling ligands that bind to them, and secreted morphogens that diffuse through the embryo. All of these help to define the fate of undifferentiated cells in the embryo. Together, they generate the patterns in time and space which shape the embryo, and ultimately form the body plan of the organism. Among the most important toolkit genes are the Hox genes. These transcription factors contain the homeobox protein-binding DNA motif, also found in other toolkit genes, and create the basic pattern of the body along its front-to-back axis. Hox genes determine where repeating parts, such as the many vertebrae of snakes, will grow in a developing embryo or larva. Pax-6, already mentioned, is a classic toolkit gene. Homeobox genes are also found in plants, implying they are common to all eukaryotes.

The embryo's regulatory networks

The protein products of the regulatory toolkit are reused not by duplication and modification, but by a complex mosaic of pleiotropy, being applied unchanged in many independent developmental processes, giving pattern to many dissimilar body structures. The loci of these pleiotropic toolkit genes have large, complicated and modular cis-regulatory elements. For example, while a non-pleiotropic rhodopsin gene in the fruit fly has a cis-regulatory element just a few hundred base pairs long, the pleiotropic eyeless cis-regulatory region contains 6 cis-regulatory elements in over 7000 base pairs. The regulatory networks involved are often very large. Each regulatory protein controls ``scores to hundreds'' of cis-regulatory elements. For instance, 67 fruit fly transcription factors controlled on average 124 target genes each. All this complexity enables genes involved in the development of the embryo to be switched on and off at exactly the right times and in exactly the right places. Some of these genes are structural, directly forming enzymes, tissues and organs of the embryo. But many others are themselves regulatory genes, so what is switched on is often a precisely-timed cascade of switching, involving turning on one developmental process after another in the developing embryo.

Such a cascading regulatory network has been studied in detail in the development of the fruit fly embryo. The young embryo is oval in shape, like a rugby ball. A small number of genes produce messenger RNAs that set up concentration gradients along the long axis of the embryo. In the early embryo, the bicoid and hunchback genes are at high concentration near the anterior end, and give pattern to the future head and thorax; the caudal and nanos genes are at high concentration near the posterior end, and give pattern to the hindmost abdominal segments. The effects of these genes interact; for instance, the Bicoid protein blocks the translation of caudal's messeger RNA, so the Caudal protein concentration becomes low at the anterior end. Caudal later switches on genes which create the fly's hindmost segments, but only at the posterior end where it is most concentrated.

The Bicoid, Hunchback and Caudal proteins in turn regulate the transcription of gap genes such as giant, knirps, Krüppel, and tailless in a striped pattern, creating the first level of structures that will become segments. The proteins from these in turn control the pair-rule genes, which in the next stage set up 7 bands across the embryo's long axis. Finally, the segment polarity genes such as engrailed split each of the 7 bands into two, creating 14 future segments.

This process explains the accurate conservation of toolkit gene sequences, which has resulted in deep homology and functional equivalence of toolkit proteins in dissimilar animals (seen, for example, when a mouse protein controls fruit fly development). The interactions of transcription factors and cis-regulatory elements, or of signalling proteins and receptors, become locked in through multiple usages, making almost any mutation deleterious and hence eliminated by natural selection.


