\chapter{Techniques of Molecular Genetics}\label{techniques-of-molecular-genetics}

\hypertarget{polymerase-chain-reaction}{%
\section{Polymerase chain reaction}\label{polymerase-chain-reaction}}

Polymerase chain reaction (PCR) is a method widely used in molecular biology to make several copies of a specific DNA segment. Using PCR, copies of DNA sequences are exponentially amplified to generate thousands to millions of more copies of that particular DNA segment. PCR is now a common and often indispensable technique used in medical laboratory and clinical laboratory research for a broad variety of applications including biomedical research and criminal forensics. The vast majority of PCR methods rely on thermal cycling. Thermal cycling exposes reactants to repeated cycles of heating and cooling to permit different temperature-dependent reactions -- specifically, DNA melting and enzyme-driven DNA replication. PCR employs two main reagents -- primers (which are short single strand DNA fragments known as oligonucleotides that are a complementary sequence to the target DNA region) and a DNA polymerase. In the first step of PCR, the two strands of the DNA double helix are physically separated at a high temperature in a process called Nucleic acid denaturation. In the second step, the temperature is lowered and the primers bind to the complementary sequences of DNA. The two DNA strands then become templates for DNA polymerase to enzymatically assemble a new DNA strand from free nucleotides, the building blocks of DNA. As PCR progresses, the DNA generated is itself used as a template for replication, setting in motion a chain reaction in which the original DNA template is exponentially amplified.

Almost all PCR applications employ a heat-stable DNA polymerase, such as Taq polymerase, an enzyme originally isolated from the thermophilic bacterium Thermus aquaticus. If the polymerase used was heat-susceptible, it would denature under the high temperatures of the denaturation step. Before the use of Taq polymerase, DNA polymerase had to be manually added every cycle, which was a tedious and costly process.

Applications of the technique include DNA cloning for sequencing, gene cloning and manipulation, gene mutagenesis; construction of DNA-based phylogenies, or functional analysis of genes; diagnosis and monitoring of hereditary diseases; amplification of ancient DNA; analysis of genetic fingerprints for DNA profiling (for example, in forensic science and parentage testing); and detection of pathogens in nucleic acid tests for the diagnosis of infectious diseases.

PCR amplifies a specific region of a DNA strand (the DNA target). Most PCR methods amplify DNA fragments of between 0.1 and 10 kilo base pairs (kbp) in length, although some techniques allow for amplification of fragments up to 40 kbp. The amount of amplified product is determined by the available substrates in the reaction, which become limiting as the reaction progresses.

A basic PCR set-up requires several components and reagents, including a DNA template that contains the DNA target region to amplify; a DNA polymerase; an enzyme that polymerizes new DNA strands; heat-resistant Taq polymerase is especially common, as it is more likely to remain intact during the high-temperature DNA denaturation process; two DNA primers that are complementary to the 3' (three prime) ends of each of the sense and anti-sense strands of the DNA target (DNA polymerase can only bind to and elongate from a double-stranded region of DNA; without primers there is no double-stranded initiation site at which the polymerase can bind); specific primers that are complementary to the DNA target region are selected beforehand, and are often custom-made in a laboratory or purchased from commercial biochemical suppliers; deoxynucleoside triphosphates, or dNTPs (sometimes called ``deoxynucleotide triphosphates''; nucleotides containing triphosphate groups), the building blocks from which the DNA polymerase synthesizes a new DNA strand; a buffer solution providing a suitable chemical environment for optimum activity and stability of the DNA polymerase; bivalent cations, typically magnesium (Mg) or manganese (Mn) ions; Mg2+ is the most common, but Mn2+ can be used for PCR-mediated DNA mutagenesis, as a higher Mn2+ concentration increases the error rate during DNA synthesis; and monovalent cations, typically potassium (K) ions

The reaction is commonly carried out in a volume of 10--200 μL in small reaction tubes (0.2--0.5 mL volumes) in a thermal cycler. The thermal cycler heats and cools the reaction tubes to achieve the temperatures required at each step of the reaction (see below). Many modern thermal cyclers make use of the Peltier effect, which permits both heating and cooling of the block holding the PCR tubes simply by reversing the electric current. Thin-walled reaction tubes permit favorable thermal conductivity to allow for rapid thermal equilibration. Most thermal cyclers have heated lids to prevent condensation at the top of the reaction tube. Older thermal cyclers lacking a heated lid require a layer of oil on top of the reaction mixture or a ball of wax inside the tube.

Procedure
Typically, PCR consists of a series of 20--40 repeated temperature changes, called thermal cycles, with each cycle commonly consisting of two or three discrete temperature steps (see figure below). The cycling is often preceded by a single temperature step at a very high temperature (\textgreater{}90 °C (194 °F)), and followed by one hold at the end for final product extension or brief storage. The temperatures used and the length of time they are applied in each cycle depend on a variety of parameters, including the enzyme used for DNA synthesis, the concentration of bivalent ions and dNTPs in the reaction, and the melting temperature (Tm) of the primers. The individual steps common to most PCR methods are as follows:

Initialization: This step is only required for DNA polymerases that require heat activation by hot-start PCR. It consists of heating the reaction chamber to a temperature of 94--96 °C (201--205 °F), or 98 °C (208 °F) if extremely thermostable polymerases are used, which is then held for 1--10 minutes.
Denaturation: This step is the first regular cycling event and consists of heating the reaction chamber to 94--98 °C (201--208 °F) for 20--30 seconds. This causes DNA melting, or denaturation, of the double-stranded DNA template by breaking the hydrogen bonds between complementary bases, yielding two single-stranded DNA molecules.
Annealing: In the next step, the reaction temperature is lowered to 50--65 °C (122--149 °F) for 20--40 seconds, allowing annealing of the primers to each of the single-stranded DNA templates. Two different primers are typically included in the reaction mixture: one for each of the two single-stranded complements containing the target region. The primers are single-stranded sequences themselves, but are much shorter than the length of the target region, complementing only very short sequences at the 3' end of each strand.
It is critical to determine a proper temperature for the annealing step because efficiency and specificity are strongly affected by the annealing temperature. This temperature must be low enough to allow for hybridization of the primer to the strand, but high enough for the hybridization to be specific, i.e., the primer should bind only to a perfectly complementary part of the strand, and nowhere else. If the temperature is too low, the primer may bind imperfectly. If it is too high, the primer may not bind at all. A typical annealing temperature is about 3--5 °C below the Tm of the primers used. Stable hydrogen bonds between complementary bases are formed only when the primer sequence very closely matches the template sequence. During this step, the polymerase binds to the primer-template hybrid and begins DNA formation.
Extension/elongation: The temperature at this step depends on the DNA polymerase used; the optimum activity temperature for the thermostable DNA polymerase of Taq (Thermus aquaticus) polymerase is approximately 75--80 °C (167--176 °F), though a temperature of 72 °C (162 °F) is commonly used with this enzyme. In this step, the DNA polymerase synthesizes a new DNA strand complementary to the DNA template strand by adding free dNTPs from the reaction mixture that are complementary to the template in the 5'-to-3' direction, condensing the 5'-phosphate group of the dNTPs with the 3'-hydroxy group at the end of the nascent (elongating) DNA strand. The precise time required for elongation depends both on the DNA polymerase used and on the length of the DNA target region to amplify. As a rule of thumb, at their optimal temperature, most DNA polymerases polymerize a thousand bases per minute. Under optimal conditions (i.e., if there are no limitations due to limiting substrates or reagents), at each extension/elongation step, the number of DNA target sequences is doubled. With each successive cycle, the original template strands plus all newly generated strands become template strands for the next round of elongation, leading to exponential (geometric) amplification of the specific DNA target region.
The processes of denaturation, annealing and elongation constitute a single cycle. Multiple cycles are required to amplify the DNA target to millions of copies. The formula used to calculate the number of DNA copies formed after a given number of cycles is 2n, where n is the number of cycles. Thus, a reaction set for 30 cycles results in 230, or 1073741824, copies of the original double-stranded DNA target region.
Final elongation: This single step is optional, but is performed at a temperature of 70--74 °C (158--165 °F) (the temperature range required for optimal activity of most polymerases used in PCR) for 5--15 minutes after the last PCR cycle to ensure that any remaining single-stranded DNA is fully elongated.
Final hold: The final step cools the reaction chamber to 4--15 °C (39--59 °F) for an indefinite time, and may be employed for short-term storage of the PCR products.

A 1971 paper in the Journal of Molecular Biology by Kjell Kleppe {[}no{]} and co-workers in the laboratory of H. Gobind Khorana first described a method of using an enzymatic assay to replicate a short DNA template with primers in vitro. However, this early manifestation of the basic PCR principle did not receive much attention at the time and the invention of the polymerase chain reaction in 1983 is generally credited to Kary Mullis.

When Mullis developed the PCR in 1983, he was working in Emeryville, California for Cetus Corporation, one of the first biotechnology companies, where he was responsible for synthesizing short chains of DNA. Mullis has written that he first conceived the idea for PCR while cruising along the Pacific Coast Highway one night in his car. He was playing in his mind with a new way of analyzing changes (mutations) in DNA when he realized that he had instead invented a method of amplifying any DNA region through repeated cycles of duplication driven by DNA polymerase. In Scientific American, Mullis summarized the procedure: ``Beginning with a single molecule of the genetic material DNA, the PCR can generate 100 billion similar molecules in an afternoon. The reaction is easy to execute. It requires no more than a test tube, a few simple reagents, and a source of heat.'' DNA fingerprinting was first used for paternity testing in 1988.

Mullis was awarded the Nobel Prize in Chemistry in 1993 for his invention, seven years after he and his colleagues at Cetus first put his proposal to practice. Mullis's 1985 paper with R. K. Saiki and H. A. Erlich, ``Enzymatic Amplification of β-globin Genomic Sequences and Restriction Site Analysis for Diagnosis of Sickle Cell Anemia''---the polymerase chain reaction invention (PCR) -- was honored by a Citation for Chemical Breakthrough Award from the Division of History of Chemistry of the American Chemical Society in 2017.

Some controversies have remained about the intellectual and practical contributions of other scientists to Mullis' work, and whether he had been the sole inventor of the PCR principle (see below).

At the core of the PCR method is the use of a suitable DNA polymerase able to withstand the high temperatures of \textgreater{}90 °C (194 °F) required for separation of the two DNA strands in the DNA double helix after each replication cycle. The DNA polymerases initially employed for in vitro experiments presaging PCR were unable to withstand these high temperatures. So the early procedures for DNA replication were very inefficient and time-consuming, and required large amounts of DNA polymerase and continuous handling throughout the process.

The discovery in 1976 of Taq polymerase---a DNA polymerase purified from the thermophilic bacterium, Thermus aquaticus, which naturally lives in hot (50 to 80 °C (122 to 176 °F)) environments such as hot springs---paved the way for dramatic improvements of the PCR method. The DNA polymerase isolated from T. aquaticus is stable at high temperatures remaining active even after DNA denaturation, thus obviating the need to add new DNA polymerase after each cycle. This allowed an automated thermocycler-based process for DNA amplification.

\hypertarget{dna-sequencing}{%
\section{DNA Sequencing}\label{dna-sequencing}}

DNA sequencing is the process of determining the nucleic acid sequence -- the order of nucleotides in DNA. It includes any method or technology that is used to determine the order of the four bases: adenine, guanine, cytosine, and thymine. The advent of rapid DNA sequencing methods has greatly accelerated biological and medical research and discovery.

Knowledge of DNA sequences has become indispensable for basic biological research, and in numerous applied fields such as medical diagnosis, biotechnology, forensic biology, virology and biological systematics. Comparing healthy and mutated DNA sequences can diagnose different diseases including various caners, characterize antibody repertoire, and can be used to guide patient treatment. Having a quick way to sequence DNA allows for faster and more individualized medical care to be administered, and for more organisms to be identified and cataloged.

The rapid speed of sequencing attained with modern DNA sequencing technology has been instrumental in the sequencing of complete DNA sequences, or genomes, of numerous types and species of life, including the human genome and other complete DNA sequences of many animal, plant, and microbial species.

An example of the results of automated chain-termination DNA sequencing.
The first DNA sequences were obtained in the early 1970s by academic researchers using laborious methods based on two-dimensional chromatography. Following the development of fluorescence-based sequencing methods with a DNA sequencer, DNA sequencing has become easier and orders of magnitude faster.

Early DNA sequencing methods
The first method for determining DNA sequences involved a location-specific primer extension strategy established by Ray Wu at Cornell University in 1970. DNA polymerase catalysis and specific nucleotide labeling, both of which figure prominently in current sequencing schemes, were used to sequence the cohesive ends of lambda phage DNA. Between 1970 and 1973, Wu, R Padmanabhan and colleagues demonstrated that this method can be employed to determine any DNA sequence using synthetic location-specific primers. Frederick Sanger then adopted this primer-extension strategy to develop more rapid DNA sequencing methods at the MRC Centre, Cambridge, UK and published a method for ``DNA sequencing with chain-terminating inhibitors'' in 1977. Walter Gilbert and Allan Maxam at Harvard also developed sequencing methods, including one for ``DNA sequencing by chemical degradation''. In 1973, Gilbert and Maxam reported the sequence of 24 basepairs using a method known as wandering-spot analysis. Advancements in sequencing were aided by the concurrent development of recombinant DNA technology, allowing DNA samples to be isolated from sources other than viruses.

Sequencing of full genomes

The first full DNA genome to be sequenced was that of bacteriophage φX174 in 1977. Medical Research Council scientists deciphered the complete DNA sequence of the Epstein-Barr virus in 1984, finding it contained 172,282 nucleotides. Completion of the sequence marked a significant turning point in DNA sequencing because it was achieved with no prior genetic profile knowledge of the virus.

A non-radioactive method for transferring the DNA molecules of sequencing reaction mixtures onto an immobilizing matrix during electrophoresis was developed by Pohl and co-workers in the early 1980s. Followed by the commercialization of the DNA sequencer ``Direct-Blotting-Electrophoresis-System GATC 1500'' by GATC Biotech, which was intensively used in the framework of the EU genome-sequencing programme, the complete DNA sequence of the yeast Saccharomyces cerevisiae chromosome II. Leroy E. Hood's laboratory at the California Institute of Technology announced the first semi-automated DNA sequencing machine in 1986. This was followed by Applied Biosystems' marketing of the first fully automated sequencing machine, the ABI 370, in 1987 and by Dupont's Genesis 2000 which used a novel fluorescent labeling technique enabling all four dideoxynucleotides to be identified in a single lane. By 1990, the U.S. National Institutes of Health (NIH) had begun large-scale sequencing trials on Mycoplasma capricolum, Escherichia coli, Caenorhabditis elegans, and Saccharomyces cerevisiae at a cost of US\$ 0.75 per base. Meanwhile, sequencing of human cDNA sequences called expressed sequence tags began in Craig Venter's lab, an attempt to capture the coding fraction of the human genome. In 1995, Venter, Hamilton Smith, and colleagues at The Institute for Genomic Research (TIGR) published the first complete genome of a free-living organism, the bacterium Haemophilus influenzae. The circular chromosome contains 1,830,137 bases and its publication in the journal Science marked the first published use of whole-genome shotgun sequencing, eliminating the need for initial mapping efforts.

By 2001, shotgun sequencing methods had been used to produce a draft sequence of the human genome.

High-throughput sequencing (HTS) methods
Several new methods for DNA sequencing were developed in the mid to late 1990s and were implemented in commercial DNA sequencers by the year 2000. Together these were called the ``next-generation'' or ``second-generation'' sequencing (NGS) methods, in order to distinguish them from the aforementioned earlier methods, like Sanger Sequencing. In contrast to the first generation of sequencing, NGS technology is typically characterized by being highly scalable, allowing the entire genome to be sequenced at once. Usually, this is accomplished by fragmenting the genome into small pieces, randomly sampling for a fragment, and sequencing it using one of a variety of technologies, such as those described below. An entire genome is possible because multiple fragments are sequenced at once (giving it the name ``massively parallel'' sequencing) in an automated process.

NGS technology has tremendously empowered researchers to look for insights into health, anthropologists to investigate human origins, and is catalyzing the ``Personalized Medicine'' movement. However, it has also opened the door to more room for error. There are many software tools to carry out the computational analysis of NGS data, each with its own algorithm. Even the parameters within one software package can change the outcome of the analysis. In addition, the large quantities of data produced by DNA sequencing have also required development of new methods and programs for sequence analysis. Several efforts to develop standards in the NGS field have been attempted to address these challenges, most of which have been small-scale efforts arising from individual labs. Most recently, a large, organized, FDA-funded effort has culminated in the BioCompute standard.

On 26 October 1990, Roger Tsien, Pepi Ross, Margaret Fahnestock and Allan J Johnston filed a patent describing stepwise (``base-by-base'') sequencing with removable 3' blockers on DNA arrays (blots and single DNA molecules). In 1996, Pål Nyrén and his student Mostafa Ronaghi at the Royal Institute of Technology in Stockholm published their method of pyrosequencing.

On 1 April 1997, Pascal Mayer and Laurent Farinelli submitted patents to the World Intellectual Property Organization describing DNA colony sequencing. The DNA sample preparation and random surface-polymerase chain reaction (PCR) arraying methods described in this patent, coupled to Roger Tsien et al.'s ``base-by-base'' sequencing method, is now implemented in Illumina's Hi-Seq genome sequencers.

In 1998, Phil Green and Brent Ewing of the University of Washington described their phred quality score for sequencer data analysis, a landmark analysis technique that gained widespread adoption, and which is still the most common metric for assessing the accuracy of a sequencing platform.

Lynx Therapeutics published and marketed massively parallel signature sequencing (MPSS), in 2000. This method incorporated a parallelized, adapter/ligation-mediated, bead-based sequencing technology and served as the first commercially available ``next-generation'' sequencing method, though no DNA sequencers were sold to independent laboratories.

Basic methods
Maxam-Gilbert sequencing
Main article: Maxam-Gilbert sequencing
Allan Maxam and Walter Gilbert published a DNA sequencing method in 1977 based on chemical modification of DNA and subsequent cleavage at specific bases. Also known as chemical sequencing, this method allowed purified samples of double-stranded DNA to be used without further cloning. This method's use of radioactive labeling and its technical complexity discouraged extensive use after refinements in the Sanger methods had been made.

Maxam-Gilbert sequencing requires radioactive labeling at one 5' end of the DNA and purification of the DNA fragment to be sequenced. Chemical treatment then generates breaks at a small proportion of one or two of the four nucleotide bases in each of four reactions (G, A+G, C, C+T). The concentration of the modifying chemicals is controlled to introduce on average one modification per DNA molecule. Thus a series of labeled fragments is generated, from the radiolabeled end to the first ``cut'' site in each molecule. The fragments in the four reactions are electrophoresed side by side in denaturing acrylamide gels for size separation. To visualize the fragments, the gel is exposed to X-ray film for autoradiography, yielding a series of dark bands each corresponding to a radiolabeled DNA fragment, from which the sequence may be inferred.

Chain-termination methods
Main article: Sanger sequencing
The chain-termination method developed by Frederick Sanger and coworkers in 1977 soon became the method of choice, owing to its relative ease and reliability. When invented, the chain-terminator method used fewer toxic chemicals and lower amounts of radioactivity than the Maxam and Gilbert method. Because of its comparative ease, the Sanger method was soon automated and was the method used in the first generation of DNA sequencers.

Sanger sequencing is the method which prevailed from the 1980s until the mid-2000s. Over that period, great advances were made in the technique, such as fluorescent labelling, capillary electrophoresis, and general automation. These developments allowed much more efficient sequencing, leading to lower costs. The Sanger method, in mass production form, is the technology which produced the first human genome in 2001, ushering in the age of genomics. However, later in the decade, radically different approaches reached the market, bringing the cost per genome down from US\$ 100 million in 2001 to US\$ 10,000 in 2011.

Advanced methods and de novo sequencing

Large-scale sequencing often aims at sequencing very long DNA pieces, such as whole chromosomes, although large-scale sequencing can also be used to generate very large numbers of short sequences, such as found in phage display. For longer targets such as chromosomes, common approaches consist of cutting (with restriction enzymes) or shearing (with mechanical forces) large DNA fragments into shorter DNA fragments. The fragmented DNA may then be cloned into a DNA vector and amplified in a bacterial host such as Escherichia coli. Short DNA fragments purified from individual bacterial colonies are individually sequenced and assembled electronically into one long, contiguous sequence. Studies have shown that adding a size selection step to collect DNA fragments of uniform size can improve sequencing efficiency and accuracy of the genome assembly. In these studies, automated sizing has proven to be more reproducible and precise than manual gel sizing.

The term ``de novo sequencing'' specifically refers to methods used to determine the sequence of DNA with no previously known sequence. De novo translates from Latin as ``from the beginning''. Gaps in the assembled sequence may be filled by primer walking. The different strategies have different tradeoffs in speed and accuracy; shotgun methods are often used for sequencing large genomes, but its assembly is complex and difficult, particularly with sequence repeats often causing gaps in genome assembly.

Most sequencing approaches use an in vitro cloning step to amplify individual DNA molecules, because their molecular detection methods are not sensitive enough for single molecule sequencing. Emulsion PCR isolates individual DNA molecules along with primer-coated beads in aqueous droplets within an oil phase. A polymerase chain reaction (PCR) then coats each bead with clonal copies of the DNA molecule followed by immobilization for later sequencing. Emulsion PCR is used in the methods developed by Marguilis et al. (commercialized by 454 Life Sciences), Shendure and Porreca et al. (also known as ``polony sequencing'') and SOLiD sequencing, (developed by Agencourt, later Applied Biosystems, now Life Technologies). Emulsion PCR is also used in the GemCode and Chromium platforms developed by 10x Genomics.

Shotgun sequencing
Main article: Shotgun sequencing
Shotgun sequencing is a sequencing method designed for analysis of DNA sequences longer than 1000 base pairs, up to and including entire chromosomes. This method requires the target DNA to be broken into random fragments. After sequencing individual fragments, the sequences can be reassembled on the basis of their overlapping regions.

Bridge PCR
Another method for in vitro clonal amplification is bridge PCR, in which fragments are amplified upon primers attached to a solid surface and form ``DNA colonies'' or ``DNA clusters''. This method is used in the Illumina Genome Analyzer sequencers. Single-molecule methods, such as that developed by Stephen Quake's laboratory (later commercialized by Helicos) are an exception: they use bright fluorophores and laser excitation to detect base addition events from individual DNA molecules fixed to a surface, eliminating the need for molecular amplification.

High-throughput methods

High-throughput, or next-generation,{[}nt 1{]} sequencing applies to genome sequencing, genome resequencing, transcriptome profiling (RNA-Seq), DNA-protein interactions (ChIP-sequencing), and epigenome characterization. Resequencing is necessary, because the genome of a single individual of a species will not indicate all of the genome variations among other individuals of the same species.

The high demand for low-cost sequencing has driven the development of high-throughput sequencing technologies that parallelize the sequencing process, producing thousands or millions of sequences concurrently. High-throughput sequencing technologies are intende to lower the cost of DNA sequencing beyond what is possible with standard dye-terminator methods. In ultra-high-throughput sequencing as many as 500,000 sequencing-by-synthesis operations may be run in parallel. Such technologies led to the ability to sequence an entire human genome in as little as one day. As of 2019, corporate leaders in the development of high-throughput sequencing products included Illumina, Qiagen and ThermoFisher Scientific.

Massively parallel signature sequencing (MPSS)
The first of the high-throughput sequencing technologies, massively parallel signature sequencing (or MPSS), was developed in the 1990s at Lynx Therapeutics, a company founded in 1992 by Sydney Brenner and Sam Eletr. MPSS was a bead-based method that used a complex approach of adapter ligation followed by adapter decoding, reading the sequence in increments of four nucleotides. This method made it susceptible to sequence-specific bias or loss of specific sequences. Because the technology was so complex, MPSS was only performed `in-house' by Lynx Therapeutics and no DNA sequencing machines were sold to independent laboratories. Lynx Therapeutics merged with Solexa (later acquired by Illumina) in 2004, leading to the development of sequencing-by-synthesis, a simpler approach acquired from Manteia Predictive Medicine, which rendered MPSS obsolete. However, the essential properties of the MPSS output were typical of later high-throughput data types, including hundreds of thousands of short DNA sequences. In the case of MPSS, these were typically used for sequencing cDNA for measurements of gene expression levels.

Polony sequencing
Main article: Polony sequencing
The polony sequencing method, developed in the laboratory of George M. Church at Harvard, was among the first high-throughput sequencing systems and was used to sequence a full E. coli genome in 2005. It combined an in vitro paired-tag library with emulsion PCR, an automated microscope, and ligation-based sequencing chemistry to sequence an E. coli genome at an accuracy of \textgreater{}99.9999\% and a cost approximately 1/9 that of Sanger sequencing. The technology was licensed to Agencourt Biosciences, subsequently spun out into Agencourt Personal Genomics, and eventually incorporated into the Applied Biosystems SOLiD platform. Applied Biosystems was later acquired by Life Technologies, now part of Thermo Fisher Scientific.

454 pyrosequencing
Main article: 454 Life Sciences § Technology
A parallelized version of pyrosequencing was developed by 454 Life Sciences, which has since been acquired by Roche Diagnostics. The method amplifies DNA inside water droplets in an oil solution (emulsion PCR), with each droplet containing a single DNA template attached to a single primer-coated bead that then forms a clonal colony. The sequencing machine contains many picoliter-volume wells each containing a single bead and sequencing enzymes. Pyrosequencing uses luciferase to generate light for detection of the individual nucleotides added to the nascent DNA, and the combined data are used to generate sequence reads. This technology provides intermediate read length and price per base compared to Sanger sequencing on one end and Solexa and SOLiD on the other.

Illumina (Solexa) sequencing

Solexa, now part of Illumina, was founded by Shankar Balasubramanian and David Klenerman in 1998, and developed a sequencing method based on reversible dye-terminators technology, and engineered polymerases. The reversible terminated chemistry concept was invented by Bruno Canard and Simon Sarfati at the Pasteur Institute in Paris. It was developed internally at Solexa by those named on the relevant patents. In 2004, Solexa acquired the company Manteia Predictive Medicine in order to gain a massively parallel sequencing technology invented in 1997 by Pascal Mayer and Laurent Farinelli. It is based on ``DNA clusters'' or ``DNA colonies'', which involves the clonal amplification of DNA on a surface. The cluster technology was co-acquired with Lynx Therapeutics of California. Solexa Ltd.~later merged with Lynx to form Solexa Inc.

In this method, DNA molecules and primers are first attached on a slide or flow cell and amplified with polymerase so that local clonal DNA colonies, later coined ``DNA clusters'', are formed. To determine the sequence, four types of reversible terminator bases (RT-bases) are added and non-incorporated nucleotides are washed away. A camera takes images of the fluorescently labeled nucleotides. Then the dye, along with the terminal 3' blocker, is chemically removed from the DNA, allowing for the next cycle to begin. Unlike pyrosequencing, the DNA chains are extended one nucleotide at a time and image acquisition can be performed at a delayed moment, allowing for very large arrays of DNA colonies to be captured by sequential images taken from a single camera.

Decoupling the enzymatic reaction and the image capture allows for optimal throughput and theoretically unlimited sequencing capacity. With an optimal configuration, the ultimately reachable instrument throughput is thus dictated solely by the analog-to-digital conversion rate of the camera, multiplied by the number of cameras and divided by the number of pixels per DNA colony required for visualizing them optimally (approximately 10 pixels/colony). In 2012, with cameras operating at more than 10 MHz A/D conversion rates and available optics, fluidics and enzymatics, throughput can be multiples of 1 million nucleotides/second, corresponding roughly to 1 human genome equivalent at 1x coverage per hour per instrument, and 1 human genome re-sequenced (at approx. 30x) per day per instrument (equipped with a single camera).

Combinatorial probe anchor synthesis (cPAS)
This method is an upgraded modification to combinatorial probe anchor ligation technology (cPAL) described by Complete Genomics which has since become part of Chinese genomics company BGI in 2013. The two companies have refined the technology to allow for longer read lengths, reaction time reductions and faster time to results. In addition, data are now generated as contiguous full-length reads in the standard FASTQ file format and can be used as-is in most short-read-based bioinformatics analysis pipelines.{[}citation needed{]}

The two technologies that form the basis for this high-throughput sequencing technology are DNA nanoballs (DNB) and patterned arrays for nanoball attachment to a solid surface. DNA nanoballs are simply formed by denaturing double stranded, adapter ligated libraries and ligating the forward strand only to a splint oligonucleotide to form a ssDNA circle. Faithful copies of the circles containing the DNA insert are produced utilizing Rolling Circle Amplification that generates approximately 300--500 copies. The long strand of ssDNA folds upon itself to produce a three-dimensional nanoball structure that is approximately 220 nm in diameter. Making DNBs replaces the need to generate PCR copies of the library on the flow cell and as such can remove large proportions of duplicate reads, adapter-adapter ligations and PCR induced errors.

The patterned array of positively charged spots is fabricated through photolithography and etching techniques followed by chemical modification to generate a sequencing flow cell. Each spot on the flow cell is approximately 250 nm in diameter, are separated by 700 nm (centre to centre) and allows easy attachment of a single negatively charged DNB to the flow cell and thus reducing under or over-clustering on the flow cell.{[}citation needed{]}

Sequencing is then performed by addition of an oligonucleotide probe that attaches in combination to specific sites within the DNB. The probe acts as an anchor that then allows one of four single reversibly inactivated, labelled nucleotides to bind after flowing across the flow cell. Unbound nucleotides are washed away before laser excitation of the attached labels then emit fluorescence and signal is captured by cameras that is converted to a digital output for base calling. The attached base has its terminator and label chemically cleaved at completion of the cycle. The cycle is repeated with another flow of free, labelled nucleotides across the flow cell to allow the next nucleotide to bind and have its signal captured. This process is completed a number of times (usually 50 to 300 times) to determine the sequence of the inserted piece of DNA at a rate of approximately 40 million nucleotides per second as of 2018.{[}citation needed{]}

SOLiD sequencing

Applied Biosystems' (now a Life Technologies brand) SOLiD technology employs sequencing by ligation. Here, a pool of all possible oligonucleotides of a fixed length are labeled according to the sequenced position. Oligonucleotides are annealed and ligated; the preferential ligation by DNA ligase for matching sequences results in a signal informative of the nucleotide at that position. Before sequencing, the DNA is amplified by emulsion PCR. The resulting beads, each containing single copies of the same DNA molecule, are deposited on a glass slide. The result is sequences of quantities and lengths comparable to Illumina sequencing. This sequencing by ligation method has been reported to have some issue sequencing palindromic sequences.

Ion Torrent semiconductor sequencing
Main article: Ion semiconductor sequencing
Ion Torrent Systems Inc. (now owned by Life Technologies) developed a system based on using standard sequencing chemistry, but with a novel, semiconductor-based detection system. This method of sequencing is based on the detection of hydrogen ions that are released during the polymerisation of DNA, as opposed to the optical methods used in other sequencing systems. A microwell containing a template DNA strand to be sequenced is flooded with a single type of nucleotide. If the introduced nucleotide is complementary to the leading template nucleotide it is incorporated into the growing complementary strand. This causes the release of a hydrogen ion that triggers a hypersensitive ion sensor, which indicates that a reaction has occurred. If homopolymer repeats are present in the template sequence, multiple nucleotides will be incorporated in a single cycle. This leads to a corresponding number of released hydrogens and a proportionally higher electronic signal.

DNA nanoball sequencing
Main article: DNA nanoball sequencing
DNA nanoball sequencing is a type of high throughput sequencing technology used to determine the entire genomic sequence of an organism. The company Complete Genomics uses this technology to sequence samples submitted by independent researchers. The method uses rolling circle replication to amplify small fragments of genomic DNA into DNA nanoballs. Unchained sequencing by ligation is then used to determine the nucleotide sequence. This method of DNA sequencing allows large numbers of DNA nanoballs to be sequenced per run and at low reagent costs compared to other high-throughput sequencing platforms. However, only short sequences of DNA are determined from each DNA nanoball which makes mapping the short reads to a reference genome difficult. This technology has been used for multiple genome sequencing projects and is scheduled to be used for more.

Heliscope single molecule sequencing
Main article: Helicos single molecule fluorescent sequencing
Heliscope sequencing is a method of single-molecule sequencing developed by Helicos Biosciences. It uses DNA fragments with added poly-A tail adapters which are attached to the flow cell surface. The next steps involve extension-based sequencing with cyclic washes of the flow cell with fluorescently labeled nucleotides (one nucleotide type at a time, as with the Sanger method). The reads are performed by the Heliscope sequencer. The reads are short, averaging 35 bp. What made this technology especially novel was that it was the first of its class to sequence non-amplified DNA, thus preventing any read errors associated with amplification steps. In 2009 a human genome was sequenced using the Heliscope, however in 2012 the company went bankrupt.

Single molecule real time (SMRT) sequencing
Main article: Single molecule real time sequencing
SMRT sequencing is based on the sequencing by synthesis approach. The DNA is synthesized in zero-mode wave-guides (ZMWs) -- small well-like containers with the capturing tools located at the bottom of the well. The sequencing is performed with use of unmodified polymerase (attached to the ZMW bottom) and fluorescently labelled nucleotides flowing freely in the solution. The wells are constructed in a way that only the fluorescence occurring by the bottom of the well is detected. The fluorescent label is detached from the nucleotide upon its incorporation into the DNA strand, leaving an unmodified DNA strand. According to Pacific Biosciences (PacBio), the SMRT technology developer, this methodology allows detection of nucleotide modifications (such as cytosine methylation). This happens through the observation of polymerase kinetics. This approach allows reads of 20,000 nucleotides or more, with average read lengths of 5 kilobases. In 2015, Pacific Biosciences announced the launch of a new sequencing instrument called the Sequel System, with 1 million ZMWs compared to 150,000 ZMWs in the PacBio RS II instrument. SMRT sequencing is referred to as ``third-generation'' or ``long-read'' sequencing.

Nanopore DNA sequencing
Main article: Nanopore sequencing
The DNA passing through the nanopore changes its ion current. This change is dependent on the shape, size and length of the DNA sequence. Each type of the nucleotide blocks the ion flow through the pore for a different period of time. The method does not require modified nucleotides and is performed in real time. Nanopore sequencing is referred to as ``third-generation'' or ``long-read'' sequencing, along with SMRT sequencing.

Early industrial research into this method was based on a technique called `exonuclease sequencing', where the readout of electrical signals occurred as nucleotides passed by alpha(α)-hemolysin pores covalently bound with cyclodextrin. However the subsequent commercial method, `strand sequencing', sequenced DNA bases in an intact strand.

Two main areas of nanopore sequencing in development are solid state nanopore sequencing, and protein based nanopore sequencing. Protein nanopore sequencing utilizes membrane protein complexes such as α-hemolysin, MspA (Mycobacterium smegmatis Porin A) or CssG, which show great promise given their ability to distinguish between individual and groups of nucleotides. In contrast, solid-state nanopore sequencing utilizes synthetic materials such as silicon nitride and aluminum oxide and it is preferred for its superior mechanical ability and thermal and chemical stability. The fabrication method is essential for this type of sequencing given that the nanopore array can contain hundreds of pores with diameters smaller than eight nanometers.

The concept originated from the idea that single stranded DNA or RNA molecules can be electrophoretically driven in a strict linear sequence through a biological pore that can be less than eight nanometers, and can be detected given that the molecules release an ionic current while moving through the pore. The pore contains a detection region capable of recognizing different bases, with each base generating various time specific signals corresponding to the sequence of bases as they cross the pore which are then evaluated. Precise control over the DNA transport through the pore is crucial for success. Various enzymes such as exonucleases and polymerases have been used to moderate this process by positioning them near the pore's entrance.

Microfluidic Systems
There are two main microfluidic systems that are used to sequence DNA; droplet based microfluidics and digital microfluidics. Microfluidic devices solve many of the current limitations of current sequencing arrays.

Abate et al.~studied the use of droplet-based microfluidic devices for DNA sequencing. These devices have the ability to form and process picoliter sized droplets at the rate of thousands per second. The devices were created from polydimethylsiloxane (PDMS)and used Forster resonance energy transfer, FRET assays to read the sequences of DNA encompassed in the droplets. Each position on the array tested for a specific 15 base sequence.

Fair et al.~used digital microfluidic devices to study DNA pyrosequencing. Significant advantages include the portability of the device, reagent volume, speed of analysis, mass manufacturing abilities, and high throughput. This study provided a proof of concept showing that digital devices can be used for pyrosequencing; the study included using synthesis, which involves the extension of the enzymes and addition of labeled nucleotides.

Boles et al.~also studied pyrosequencing on digital microfluidic devices. They used an electro-wetting device to create, mix, and split droplets. The sequencing uses a three-enzyme protocol and DNA templates anchored with magnetic beads. The device was tested using two protocols and resulted in 100\% accuracy based on raw pyrogram levels. The advantages of these digital microfluidic devices include size, cost, and achievable levels of functional integration.

DNA sequencing research, using microfluidics, also has the ability to be applied to the sequencing of RNA, using similar droplet microfluidic techniques, such as the method, inDrops. This shows that many of these DNA sequencing techniques will be able to be applied further and be used to understand more about genomes and transcriptomes.

Methods in development
DNA sequencing methods currently under development include reading the sequence as a DNA strand transits through nanopores (a method that is now commercial but subsequent generations such as solid-state nanopores are still in development), and microscopy-based techniques, such as atomic force microscopy or transmission electron microscopy that are used to identify the positions of individual nucleotides within long DNA fragments (\textgreater{}5,000 bp) by nucleotide labeling with heavier elements (e.g., halogens) for visual detection and recording. Third generation technologies aim to increase throughput and decrease the time to result and cost by eliminating the need for excessive reagents and harnessing the processivity of DNA polymerase.

Tunnelling currents DNA sequencing
Another approach uses measurements of the electrical tunnelling currents across single-strand DNA as it moves through a channel. Depending on its electronic structure, each base affects the tunnelling current differently, allowing differentiation between different bases.

The use of tunnelling currents has the potential to sequence orders of magnitude faster than ionic current methods and the sequencing of several DNA oligomers and micro-RNA has already been achieved.

Sequencing by hybridization
Sequencing by hybridization is a non-enzymatic method that uses a DNA microarray. A single pool of DNA whose sequence is to be determined is fluorescently labeled and hybridized to an array containing known sequences. Strong hybridization signals from a given spot on the array identifies its sequence in the DNA being sequenced.

This method of sequencing utilizes binding characteristics of a library of short single stranded DNA molecules (oligonucleotides), also called DNA probes, to reconstruct a target DNA sequence. Non-specific hybrids are removed by washing and the target DNA is eluted. quired prior to sequencing. In the case of next-generation sequencing methods, library preparation is required before processing. Assessing the quality and quantity of nucleic acids both after extraction and after library preparation identifies degraded, fragmented, and low-purity samples and yields high-quality sequencing data.

Hybrids are re-arranged such that the DNA sequence can be reconstructed. The benefit of this sequencing type is its ability to capture a large number of targets with a homogenous coverage. A large number of chemicals and starting DNA is usually required. However, with the advent of solution-based hybridization, much less equipment and chemicals are necessary.

Sequencing with mass spectrometry
Mass spectrometry may be used to determine DNA sequences. Matrix-assisted laser desorption ionization time-of-flight mass spectrometry, or MALDI-TOF MS, has specifically been investigated as an alternative method to gel electrophoresis for visualizing DNA fragments. With this method, DNA fragments generated by chain-termination sequencing reactions are compared by mass rather than by size. The mass of each nucleotide is different from the others and this difference is detectable by mass spectrometry. Single-nucleotide mutations in a fragment can be more easily detected with MS than by gel electrophoresis alone. MALDI-TOF MS can more easily detect differences between RNA fragments, so researchers may indirectly sequence DNA with MS-based methods by converting it to RNA first.

The higher resolution of DNA fragments permitted by MS-based methods is of special interest to researchers in forensic science, as they may wish to find single-nucleotide polymorphisms in human DNA samples to identify individuals. These samples may be highly degraded so forensic researchers often prefer mitochondrial DNA for its higher stability and applications for lineage studies. MS-based sequencing methods have been used to compare the sequences of human mitochondrial DNA from samples in a Federal Bureau of Investigation database and from bones found in mass graves of World War I soldiers.

Early chain-termination and TOF MS methods demonstrated read lengths of up to 100 base pairs. Researchers have been unable to exceed this average read size; like chain-termination sequencing alone, MS-based DNA sequencing may not be suitable for large de novo sequencing projects. Even so, a recent study did use the short sequence reads and mass spectroscopy to compare single-nucleotide polymorphisms in pathogenic Streptococcus strains.

Microfluidic Sanger sequencing
Main article: Sanger sequencing
In microfluidic Sanger sequencing the entire thermocycling amplification of DNA fragments as well as their separation by electrophoresis is done on a single glass wafer (approximately 10 cm in diameter) thus reducing the reagent usage as well as cost. In some instances researchers have shown that they can increase the throughput of conventional sequencing through the use of microchips. Research will still need to be done in order to make this use of technology effective.

Microscopy-based techniques
Main article: Transmission electron microscopy DNA sequencing
This approach directly visualizes the sequence of DNA molecules using electron microscopy. The first identification of DNA base pairs within intact DNA molecules by enzymatically incorporating modified bases, which contain atoms of increased atomic number, direct visualization and identification of individually labeled bases within a synthetic 3,272 base-pair DNA molecule and a 7,249 base-pair viral genome has been demonstrated.

RNAP sequencing
This method is based on use of RNA polymerase (RNAP), which is attached to a polystyrene bead. One end of DNA to be sequenced is attached to another bead, with both beads being placed in optical traps. RNAP motion during transcription brings the beads in closer and their relative distance changes, which can then be recorded at a single nucleotide resolution. The sequence is deduced based on the four readouts with lowered concentrations of each of the four nucleotide types, similarly to the Sanger method. A comparison is made between regions and sequence information is deduced by comparing the known sequence regions to the unknown sequence regions.

In vitro virus high-throughput sequencing
A method has been developed to analyze full sets of protein interactions using a combination of 454 pyrosequencing and an in vitro virus mRNA display method. Specifically, this method covalently links proteins of interest to the mRNAs encoding them, then detects the mRNA pieces using reverse transcription PCRs. The mRNA may then be amplified and sequenced. The combined method was titled IVV-HiTSeq and can be performed under cell-free conditions, though its results may not be representative of in vivo conditions.

Sample preparation
The success of any DNA sequencing protocol relies upon the DNA or RNA sample extraction and preparation from the biological material of interest.

A successful DNA extraction will yield a DNA sample with long, non-degraded strands.
A successful RNA extraction will yield a RNA sample that should be converted to complementary DNA (cDNA) using reverse transcriptase---a DNA polymerase that synthesizes a complementary DNA based on existing strands of RNA in a PCR-like manner. Complementary DNA can then be processed the same way as genomic DNA.
According to the sequencing technology to be used, the samples resulting from either the DNA or the RNA extraction require further preparation. For Sanger sequencing, either cloning procedures or PCR are required.

\hypertarget{rna-seq}{%
\section{RNA-Seq}\label{rna-seq}}

RNA-Seq uses next-generation sequencing (NGS) to reveal the presence and quantity of RNA in a biological sample at a given moment, analyzing the continuously changing cellular transcriptome.

Specifically, RNA-Seq facilitates the ability to look at alternative gene spliced transcripts, post-transcriptional modifications, gene fusion, mutations/SNPs and changes in gene expression over time, or differences in gene expression in different groups or treatments. In addition to mRNA transcripts, RNA-Seq can look at different populations of RNA to include total RNA, small RNA, such as miRNA, tRNA, and ribosomal profiling. RNA-Seq can also be used to determine exon/intron boundaries and verify or amend previously annotated 5' and 3' gene boundaries. Recent advances in RNA-Seq include single cell sequencing and in situ sequencing of fixed tissue.

Prior to RNA-Seq, gene expression studies were done with hybridization-based microarrays. Issues with microarrays include cross-hybridization artifacts, poor quantification of lowly and highly expressed genes, and needing to know the sequence a priori. Because of these technical issues, transcriptomics transitioned to sequencing-based methods. These progressed from Sanger sequencing of Expressed Sequence Tag libraries, to chemical tag-based methods (e.g., serial analysis of gene expression), and finally to the current technology, next-gen sequencing of cDNA (notably RNA-Seq).

Library preparation

The general steps to prepare a complementary DNA (cDNA) library for sequencing are described below, but often vary between platforms.

\begin{enumerate}
\def\labelenumi{\arabic{enumi}.}
\tightlist
\item
  RNA Isolation: RNA is isolated from tissue and mixed with deoxyribonuclease (DNase). DNase reduces the amount of genomic DNA. The amount of RNA degradation is checked with gel and capillary electrophoresis and is used to assign an RNA integrity number to the sample. This RNA quality and the total amount of starting RNA are taken into consideration during the subsequent library preparation, sequencing, and analysis steps.
\item
  RNA selection/depletion: To analyze signals of interest, the isolated RNA can either be kept as is, filtered for RNA with 3' polyadenylated (poly(A)) tails to include only mRNA, depleted of ribosomal RNA (rRNA), and/or filtered for RNA that binds specific sequences (RNA selection and depletion methods table, below). The RNA with 3' poly(A) tails are mature, processed, coding sequences. Poly(A) selection is performed by mixing RNA with poly(T) oligomers covalently attached to a substrate, typically magnetic beads. Poly(A) selection ignores noncoding RNA and introduces 3' bias, which is avoided with the ribosomal depletion strategy. The rRNA is removed because it represents over 90\% of the RNA in a cell, which if kept would drown out other data in the transcriptome.
\item
  cDNA synthesis: RNA is reverse transcribed to cDNA because DNA is more stable and to allow for amplification (which uses DNA polymerases) and leverage more mature DNA sequencing technology. Amplification subsequent to reverse transcription results in loss of strandedness, which can be avoided with chemical labeling or single molecule sequencing. Fragmentation and size selection are performed to purify sequences that are the appropriate length for the sequencing machine. The RNA, cDNA, or both are fragmented with enzymes, sonication, or nebulizers. Fragmentation of the RNA reduces 5' bias of randomly primed-reverse transcription and the influence of primer binding sites, with the downside that the 5' and 3' ends are converted to DNA less efficiently. Fragmentation is followed by size selection, where either small sequences are removed or a tight range of sequence lengths are selected. Because small RNAs like miRNAs are lost, these are analyzed independently. The cDNA for each experiment can be indexed with a hexamer or octamer barcode, so that these experiments can be pooled into a single lane for multiplexed sequencing.
\end{enumerate}

Small RNA/non-coding RNA sequencing
When sequencing RNA other than mRNA, the library preparation is modified. The cellular RNA is selected based on the desired size range. For small RNA targets, such as miRNA, the RNA is isolated through size selection. This can be performed with a size exclusion gel, through size selection magnetic beads, or with a commercially developed kit. Once isolated, linkers are added to the 3' and 5' end then purified. The final step is cDNA generation through reverse transcription.

Direct RNA sequencing

Because converting RNA into cDNA, ligation, amplifcation, and other sample manipulations have been shown to introduce biases and artifacts that may interfere with both the proper characterization and quantification of transcripts, single molecule direct RNA sequencing has been explored by companies including Helicos (bankrupt), Oxford Nanopore Technologies, and others. This technology sequences RNA molecules directly in a massively-parallel manner.

Single-cell RNA sequencing (scRNA-Seq)
See also: Single cell sequencing
Standard methods such as microarrays and standard bulk RNA-Seq analysis analyze the expression of RNAs from large populations of cells. In mixed cell populations, these measurements may obscure critical differences between individual cells within these populations.

Single-cell RNA sequencing (scRNA-Seq) provides the expression profiles of individual cells. Although it is not possible to obtain complete information on every RNA expressed by each cell, due to the small amount of material available, patterns of gene expression can be identified through gene clustering analyses. This can uncover the existence of rare cell types within a cell population that may never have been seen before. For example, rare specialized cells in the lung called pulmonary ionocytes that express the Cystic Fibrosis Transmembrane Conductance Regulator were identified in 2018 by two groups performing scRNA-Seq on lung airway epithelia.

Experimental procedures

Current scRNA-Seq protocols involve the following steps: isolation of single cell and RNA, reverse transcription (RT), amplification, library generation and sequencing. Early methods separated individual cells into separate wells; more recent methods encapsulate individual cells in droplets in a microfluidic device, where the reverse transcription reaction takes place, converting RNAs to cDNAs. Each droplet carries a DNA ``barcode'' that uniquely labels the cDNAs derived from a single cell. Once reverse transcription is complete, the cDNAs from many cells can be mixed together for sequencing; transcripts from a particular cell are identified by the unique barcode.

Challenges for scRNA-Seq include preserving the initial relative abundance of mRNA in a cell and identifying rare transcripts. The reverse transcription step is critical as the efficiency of the RT reaction determines how much of the cell's RNA population will be eventually analyzed by the sequencer. The processivity of reverse transcriptases and the priming strategies used may affect full-length cDNA production and the generation of libraries biased toward 3' or 5' end of genes.

In the amplification step, either PCR or in vitro transcription (IVT) is currently used to amplify cDNA. One of the advantages of PCR-based methods is the ability to generate full-length cDNA. However, different PCR efficiency on particular sequences (for instance, GC content and snapback structure) may also be exponentially amplified, producing libraries with uneven coverage. On the other hand, while libraries generated by IVT can avoid PCR-induced sequence bias, specific sequences may be transcribed inefficiently, thus causing sequence drop-out or generating incomplete sequences. Several scRNA-Seq protocols have been published: Tang et al., STRT, SMART-seq, CEL-seq, RAGE-seq, , Quartz-seq. and C1-CAGE. These protocols differ in terms of strategies for reverse transcription, cDNA synthesis and amplification, and the possibility to accommodate sequence-specific barcodes (i.e.~UMIs) or the ability to process pooled samples.

In 2017, two approaches were introduced to simultaneously measure single-cell mRNA and protein expression through oligonucleotide-labeled antibodies known as REAP-seq, and CITE-seq.

Applications
scRNA-Seq is becoming widely used across biological disciplines including Development, Neurology, Oncology, Autoimmune disease, and Infectious disease.

scRNA-Seq has provided considerable insight into the development of embryos and organisms, including the worm Caenorhabditis elegans, and the regenerative planarian Schmidtea mediterranea. The first vertebrate animals to be mapped in this way were Zebrafish and Xenopus laevis. In each case multiple stages of the embryo were studied, allowing the entire process of development to be mapped on a cell-by-cell basis. Science recognized these advances as the 2018 Breakthrough of the Year.

Experimental considerations
A variety of parameters are considered when designing and conducting RNA-Seq experiments:

Tissue specificity: Gene expression varies within and between tissues, and RNA-Seq measures this mix of cell types. This may make it difficult to isolate the biological mechanism of interest. Single cell sequencing can be used to study each cell individually, mitigating this issue.
Time dependence: Gene expression changes over time, and RNA-Seq only takes a snapshot. Time course experiments can be performed to observe changes in the transcriptome.
Coverage (also known as depth): RNA harbors the same mutations observed in DNA, and detection requires deeper coverage. With high enough coverage, RNA-Seq can be used to estimate the expression of each allele. This may provide insight into phenomena such as imprinting or cis-regulatory effects. The depth of sequencing required for specific applications can be extrapolated from a pilot experiment.
Data generation artifacts (also known as technical variance): The reagents (e.g., library preparation kit), personnel involved, and type of sequencer (e.g., Illumina, Pacific Biosciences) can result in technical artifacts that might be mis-interpreted as meaningful results. As with any scientific experiment, it is prudent to conduct RNA-Seq in a well controlled setting. If this is not possible or the study is a meta-analysis, another solution is to detect technical artifacts by inferring latent variables (typically principal component analysis or factor analysis) and subsequently correcting for these variables.
Data management: A single RNA-Seq experiment in humans is usually on the order of 1 Gb. This large volume of data can pose storage issues. One solution is compressing the data using multi-purpose computational schemas (e.g., gzip) or genomics-specific schemas. The latter can be based on reference sequences or de novo. Another solution is to perform microarray experiments, which may be sufficient for hypothesis-driven work or replication studies (as opposed to exploratory research).
Analysis

Transcriptome assembly

methods are used to assign raw sequence reads to genomic features (i.e., assemble the transcriptome):

De novo: This approach does not require a reference genome to reconstruct the transcriptome, and is typically used if the genome is unknown, incomplete, or substantially altered compared to the reference. Challenges when using short reads for de novo assembly include 1) determining which reads should be joined together into contiguous sequences (contigs), 2) robustness to sequencing errors and other artifacts, and 3) computational efficiency. The primary algorithm used for de novo assembly transitioned from overlap graphs, which identify all pair-wise overlaps between reads, to de Bruijn graphs, which break reads into sequences of length k and collapse all k-mers into a hash table. Overlap graphs were used with Sanger sequencing, but do not scale well to the millions of reads generated with RNA-Seq. Examples of assemblers that use de Bruijn graphs are Velvet, Trinity, Oases, and Bridger. Paired end and long read sequencing of the same sample can mitigate the deficits in short read sequencing by serving as a template or skeleton. Metrics to assess the quality of a de novo assembly include median contig length, number of contigs and N50.

Genome guided: This approach relies on the same methods used for DNA alignment, with the additional complexity of aligning reads that cover non-continuous portions of the reference genome. These non-continuous reads are the result of sequencing spliced transcripts (see figure). Typically, alignment algorithms have two steps: 1) align short portions of the read (i.e., seed the genome), and 2) use dynamic programming to find an optimal alignment, sometimes in combination with known annotations. Software tools that use genome-guided alignment include Bowtie, TopHat (which builds on BowTie results to align splice junctions), Subread, STAR, HISAT2, Sailfish, Kallisto, and GMAP. The quality of a genome guided assembly can be measured with both 1) de novo assembly metrics (e.g., N50) and 2) comparisons to known transcript, splice junction, genome, and protein sequences using precision, recall, or their combination (e.g., F1 score). In addition, in silico assessment could be performed using simulated reads.
A note on assembly quality: The current consensus is that 1) assembly quality can vary depending on which metric is used, 2) assemblies that scored well in one species do not necessarily perform well in the other species, and 3) combining different approaches might be the most reliable.

Gene expression quantification
Expression is quantified to study cellular changes in response to external stimuli, differences between healthy and diseased states, and other research questions. Gene expression is often used as a proxy for protein abundance, but these are often not equivalent due to post transcriptional events such as RNA interference and nonsense-mediated decay.

Expression is quantified by counting the number of reads that mapped to each locus in the transcriptome assembly step. Expression can be quantified for exons or genes using contigs or reference transcript annotations. These observed RNA-Seq read counts have been robustly validated against older technologies, including expression microarrays and qPCR. Examples of tools that quantify counts are HTSeq, FeatureCounts, Rcount, maxcounts, FIXSEQ, and Cuffquant. The read counts are then converted into appropriate metrics for hypothesis testing, regressions, and other analyses. Parameters for this conversion are:

Sequencing depth/coverage: Although depth is pre-specified when conducting multiple RNA-Seq experiments, it will still vary widely between experiments. Therefore, the total number of reads generated in a single experiment is typically normalized by converting counts to fragments, reads, or counts per million mapped reads (FPM, RPM, or CPM). Sequencing depth is sometimes referred to as library size, the number of intermediary cDNA molecules in the experiment.
Gene length: Longer genes will have more fragments/reads/counts than shorter genes if transcript expression is the same. This is adjusted by dividing the FPM by the length of a gene, resulting in the metric fragments per kilobase of transcript per million mapped reads (FPKM). When looking at groups of genes across samples, FPKM is converted to transcripts per million (TPM) by dividing each FPKM by the sum of FPKMs within a sample.
Total sample RNA output: Because the same amount of RNA is extracted from each sample, samples with more total RNA will have less RNA per gene. These genes appear to have decreased expression, resulting in false positives in downstream analyses.
Variance for each gene's expression: is modeled to account for sampling error (important for genes with low read counts), increase power, and decrease false positives. Variance can be estimated as a normal, Poisson, or negative binomial distribution and is frequently decomposed into technical and biological variance.
Absolute quantification
Absolute quantification of gene expression is not possible with most RNA-Seq experiments, which quantify expression relative to all transcripts. It is possible by performing RNA-Seq with spike-ins, samples of RNA at known concentrations. After sequencing, read counts of spike-in sequences are used to determine the relationship between each gene's read counts and absolute quantities of biological fragments. In one example, this technique was used in Xenopus tropicalis embryos to determine transcription kinetics.

Differential expression
The simplest but often most powerful use of RNA-Seq is finding differences in gene expression between two or more conditions (e.g., treated vs not treated); this process is called differential expression. The outputs are frequently referred to as differentially expressed genes (DEGs) and these genes can either be up- or down-regulated (i.e., higher or lower in the condition of interest). There are many tools that perform differential expression. Most are run in R, Python, or the Unix command line. Commonly used tools include DESeq, edgeR, and voom+limma, all of which are available through R/Bioconductor. These are the common considerations when performing differential expression:

Inputs: Differential expression inputs include (1) an RNA-Seq expression matrix (M genes x N samples) and (2) a design matrix containing experimental conditions for N samples. The simplest design matrix contains one column, corresponding to labels for the condition being tested. Other covariates (also referred to as factors, features, labels, or parameters) can include batch effects, known artifacts, and any metadata that might confound or mediate gene expression. In addition to known covariates, unknown covariates can also be estimated through unsupervised machine learning approaches including principal component, surrogate variable, and PEER analyses. Hidden variable analyses are often employed for human tissue RNA-Seq data, which typically have additional artifacts not captured in the metadata (e.g., ischemic time, sourcing from multiple institutions, underlying clinical traits, collecting data across many years with many personnel).
Methods: Most tools use regression or non-parametric statistics to identify differentially expressed genes, and are either count-based (DESeq2, limma, edgeR) or assembly-based (via alignment-free quantification, sleuth, Cuffdiff, Ballgown). Following regression, most tools employ either familywise error rate (FWER) or false discovery rate (FDR) p-value adjustments to account for multiple hypotheses (in human studies, \textasciitilde{}20,000 protein-coding genes or \textasciitilde{}50,000 biotypes).
Outputs: A typical output consists of rows corresponding to the number of genes and at least three columns, each gene's log fold change (log-transform of the ratio in expression between conditions, a measure of effect size), p-value, and p-value adjusted for multiple comparisons. Genes are defined as biologically meaningful if they pass cut-offs for effect size (log fold change) and statistical significance. These cut-offs should ideally be specified a priori, but the nature of RNA-Seq experiments is often exploratory so it is difficult to predict effect sizes and pertinent cut-offs ahead of time.
Pitfalls: The raison d'etre for these complex methods is to avoid the myriad of pitfalls that can lead to statistical errors and misleading interpretations. Pitfalls include increased false positive rates (due to multiple comparisons), sample preparation artifacts, sample heterogeneity (like mixed genetic backgrounds), highly correlated samples, unaccounted for multi-level experimental designs, and poor experimental design. One notable pitfall is viewing results in Microsoft Excel. Although convenient, Excel automatically converts some gene names (SEPT1, DEC1, MARCH2) into dates or floating point numbers.
Choice of tools and benchmarking: There are numerous efforts that compare the results of these tools, with DESeq2 tending to moderately outperform other methods. As with other methods, benchmarking consists of comparing tool outputs to each other and known gold standards.
Downstream analyses for a list of differentially expressed genes come in two flavors, validating observations and making biological inferences. Owing to the pitfalls of differential expression and RNA-Seq, important observations are replicated with (1) an orthogonal method in the same samples (like real-time PCR) or (2) another, sometimes pre-registered, experiment in a new cohort. The latter helps ensure generalizability and can typically be followed up with a meta-analysis of all the pooled cohorts. The most common method for obtaining higher-level biological understanding of the results is gene set enrichment analysis, although sometimes candidate gene approaches are employed. Gene set enrichment determines if the overlap between two gene sets is statistically significant, in this case the overlap between differentially expressed genes and gene sets from known pathways/databases (e.g., Gene Ontology, KEGG, Human Phenotype Ontology) or from complementary analyses in the same data (like co-expression networks). Common tools for gene set enrichment include web interfaces (e.g., ENRICHR, g:profiler) and software packages. When evaluating enrichment results, one heuristic is to first look for enrichment of known biology as a sanity check and then expand the scope to look for novel biology.

Alternative splicing
RNA splicing is integral to eukaryotes and contributes significantly to protein regulation and diversity, occurring in \textgreater{}90\% of human genes. There are multiple alternative splicing modes: exon skipping (most common splicing mode in humans and higher eukaryotes), mutually exclusive exons, alternative donor or acceptor sites, intron retention (most common splicing mode in plants, fungi, and protozoa), alternative transcription start site (promoter), and alternative polyadenylation. One goal of RNA-Seq is to identify alternative splicing events and test if they differ between conditions. Long-read sequencing captures the full transcript and thus minimizes many of issues in estimating isoform abundance, like ambiguous read mapping. For short-read RNA-Seq, there are multiple methods to detect alternative splicing that can be classified into three main groups:

Count-based (also event-based, differential splicing): estimate exon retention. Examples are DEXSeq, MATS, and SeqGSEA.
Isoform-based (also multi-read modules, differential isoform expression): estimate isoform abundance first, and then relative abundance between conditions. Examples are Cufflinks 2 and DiffSplice.
Intron excision based: calculate alternative splicing using split reads. Examples are MAJIQ and Leafcutter.
Differential gene expression tools can also be used for differential isoform expression if isoforms are quantified ahead of time with other tools like RSEM.

Coexpression networks
Coexpression networks are data-derived representations of genes behaving in a similar way across tissues and experimental conditions. Their main purpose lies in hypothesis generation and guilt-by-association approaches for inferring functions of previously unknown genes. RNA-Seq data has been used to infer genes involved in specific pathways based on Pearson correlation, both in plants and mammals. The main advantage of RNA-Seq data in this kind of analysis over the microarray platforms is the capability to cover the entire transcriptome, therefore allowing the possibility to unravel more complete representations of the gene regulatory networks. Differential regulation of the splice isoforms of the same gene can be detected and used to predict and their biological functions. Weighted gene co-expression network analysis has been successfully used to identify co-expression modules and intramodular hub genes based on RNA seq data. Co-expression modules may correspond to cell types or pathways. Highly connected intramodular hubs can be interpreted as representatives of their respective module. An eigengene is a weighted sum of expression of all genes in a module. Eigengenes are useful biomarkers (features) for diagnosis and prognosis. Variance-Stabilizing Transformation approaches for estimating correlation coefficients based on RNA seq data have been proposed.

Variant discovery
RNA-Seq captures DNA variation, including single nucleotide variants, small insertions/deletions. and structural variation. Variant calling in RNA-Seq is similar to DNA variant calling and often employs the same tools (including SAMtools mpileup and GATK HaplotypeCaller) with adjustments to account for splicing. One unique dimension for RNA variants is allele-specific expression (ASE): the variants from only one haplotype might be preferentially expressed due to regulatory effects including imprinting and expression quantitative trait loci, and noncoding rare variants. Limitations of RNA variant identification include that it only reflects expressed regions (in humans, \textless{}5\% of the genome) and has lower quality when compared to direct DNA sequencing.

RNA editing (post-transcriptional alterations)
See also: RNA editing
Having the matching genomic and transcriptomic sequences of an individual can help detect post-transcriptional edits (RNA editing). A post-transcriptional modification event is identified if the gene's transcript has an allele/variant not observed in the genomic data.

RNA-Seq mapping of short reads over exon-exon junctions, depending on where each end maps to, it could be defined a Trans or a Cis event.
Fusion gene detection

Caused by different structural modifications in the genome, fusion genes have gained attention because of their relationship with cancer. The ability of RNA-Seq to analyze a sample's whole transcriptome in an unbiased fashion makes it an attractive tool to find these kinds of common events in cancer.

The idea follows from the process of aligning the short transcriptomic reads to a reference genome. Most of the short reads will fall within one complete exon, and a smaller but still large set would be expected to map to known exon-exon junctions. The remaining unmapped short reads would then be further analyzed to determine whether they match an exon-exon junction where the exons come from different genes. This would be evidence of a possible fusion event, however, because of the length of the reads, this could prove to be very noisy. An alternative approach is to use pair-end reads, when a potentially large number of paired reads would map each end to a different exon, giving better coverage of these events (see figure). Nonetheless, the end result consists of multiple and potentially novel combinations of genes providing an ideal starting point for further validation.

\hypertarget{dna-microarray}{%
\section{DNA microarray}\label{dna-microarray}}

A DNA microarray (also commonly known as DNA chip or biochip) is a collection of microscopic DNA spots attached to a solid surface. Scientists use DNA microarrays to measure the expression levels of large numbers of genes simultaneously or to genotype multiple regions of a genome. Each DNA spot contains picomoles (10−12 moles) of a specific DNA sequence, known as probes (or reporters or oligos). These can be a short section of a gene or other DNA element that are used to hybridize a cDNA or cRNA (also called anti-sense RNA) sample (called target) under high-stringency conditions. Probe-target hybridization is usually detected and quantified by detection of fluorophore-, silver-, or chemiluminescence-labeled targets to determine relative abundance of nucleic acid sequences in the target. The original nucleic acid arrays were macro arrays approximately 9 cm × 12 cm and the first computerized image based analysis was published in 1981. It was invented by Patrick O. Brown.

The core principle behind microarrays is hybridization between two DNA strands, the property of complementary nucleic acid sequences to specifically pair with each other by forming hydrogen bonds between complementary nucleotide base pairs. A high number of complementary base pairs in a nucleotide sequence means tighter non-covalent bonding between the two strands. After washing off non-specific bonding sequences, only strongly paired strands will remain hybridized. Fluorescently labeled target sequences that bind to a probe sequence generate a signal that depends on the hybridization conditions (such as temperature), and washing after hybridization. Total strength of the signal, from a spot (feature), depends upon the amount of target sample binding to the probes present on that spot. Microarrays use relative quantitation in which the intensity of a feature is compared to the intensity of the same feature under a different condition, and the identity of the feature is known by its position.

\hypertarget{recombinant-dna}{%
\section{Recombinant DNA}\label{recombinant-dna}}

Recombinant DNA (rDNA) molecules are DNA molecules formed by laboratory methods of genetic recombination (such as molecular cloning) to bring together genetic material from multiple sources, creating sequences that would not otherwise be found in the genome.

Recombinant DNA is the general name for a piece of DNA that has been created by combining at least two strands. Recombinant DNA is possible because DNA molecules from all organisms share the same chemical structure, and differ only in the nucleotide sequence within that identical overall structure. Recombinant DNA molecules are sometimes called chimeric DNA, because they can be made of material from two different species, like the mythical chimera. R-DNA technology uses palindromic sequences and leads to the production of sticky and blunt ends.

The DNA sequences used in the construction of recombinant DNA molecules can originate from any species. For example, plant DNA may be joined to bacterial DNA, or human DNA may be joined with fungal DNA. In addition, DNA sequences that do not occur anywhere in nature may be created by the chemical synthesis of DNA, and incorporated into recombinant molecules. Using recombinant DNA technology and synthetic DNA, literally any DNA sequence may be created and introduced into any of a very wide range of living organisms.

Proteins that can result from the expression of recombinant DNA within living cells are termed recombinant proteins. When recombinant DNA encoding a protein is introduced into a host organism, the recombinant protein is not necessarily produced. Expression of foreign proteins requires the use of specialized expression vectors and often necessitates significant restructuring by foreign coding sequences.

Recombinant DNA differs from genetic recombination in that the former results from artificial methods in the test tube, while the latter is a normal biological process that results in the remixing of existing DNA sequences in essentially all organisms.

Molecular cloning is the laboratory process used to create recombinant DNA. It is one of two most widely used methods, along with polymerase chain reaction (PCR), used to direct the replication of any specific DNA sequence chosen by the experimentalist. There are two fundamental differences between the methods. One is that molecular cloning involves replication of the DNA within a living cell, while PCR replicates DNA in the test tube, free of living cells. The other difference is that cloning involves cutting and pasting DNA sequences, while PCR amplifies by copying an existing sequence.

Formation of recombinant DNA requires a cloning vector, a DNA molecule that replicates within a living cell. Vectors are generally derived from plasmids or viruses, and represent relatively small segments of DNA that contain necessary genetic signals for replication, as well as additional elements for convenience in inserting foreign DNA, identifying cells that contain recombinant DNA, and, where appropriate, expressing the foreign DNA. The choice of vector for molecular cloning depends on the choice of host organism, the size of the DNA to be cloned, and whether and how the foreign DNA is to be expressed. The DNA segments can be combined by using a variety of methods, such as restriction enzyme/ligase cloning or Gibson assembly.

In standard cloning protocols, the cloning of any DNA fragment essentially involves seven steps: (1) Choice of host organism and cloning vector, (2) Preparation of vector DNA, (3) Preparation of DNA to be cloned, (4) Creation of recombinant DNA, (5) Introduction of recombinant DNA into the host organism, (6) Selection of organisms containing recombinant DNA, and (7) Screening for clones with desired DNA inserts and biological properties.

Expression
Main article: Protein production
Following transplantation into the host organism, the foreign DNA contained within the recombinant DNA construct may or may not be expressed. That is, the DNA may simply be replicated without expression, or it may be transcribed and translated and a recombinant protein is produced. Generally speaking, expression of a foreign gene requires restructuring the gene to include sequences that are required for producing an mRNA molecule that can be used by the host's translational apparatus (e.g.~promoter, translational initiation signal, and transcriptional terminator). Specific changes to the host organism may be made to improve expression of the ectopic gene. In addition, changes may be needed to the coding sequences as well, to optimize translation, make the protein soluble, direct the recombinant protein to the proper cellular or extracellular location, and stabilize the protein from degradation.

Properties of organisms containing recombinant DNA
In most cases, organisms containing recombinant DNA have apparently normal phenotypes. That is, their appearance, behavior and metabolism are usually unchanged, and the only way to demonstrate the presence of recombinant sequences is to examine the DNA itself, typically using a polymerase chain reaction (PCR) test. Significant exceptions exist, and are discussed below.

If the rDNA sequences encode a gene that is expressed, then the presence of RNA and/or protein products of the recombinant gene can be detected, typically using RT-PCR or western hybridization methods. Gross phenotypic changes are not the norm, unless the recombinant gene has been chosen and modified so as to generate biological activity in the host organism. Additional phenotypes that are encountered include toxicity to the host organism induced by the recombinant gene product, especially if it is over-expressed or expressed within inappropriate cells or tissues.

In some cases, recombinant DNA can have deleterious effects even if it is not expressed. One mechanism by which this happens is insertional inactivation, in which the rDNA becomes inserted into a host cell's gene. In some cases, researchers use this phenomenon to ``knock out'' genes to determine their biological function and importance. Another mechanism by which rDNA insertion into chromosomal DNA can affect gene expression is by inappropriate activation of previously unexpressed host cell genes. This can happen, for example, when a recombinant DNA fragment containing an active promoter becomes located next to a previously silent host cell gene, or when a host cell gene that functions to restrain gene expression undergoes insertional inactivation by recombinant DNA.

Uses
Recombinant DNA is widely used in biotechnology, medicine and research. Today, recombinant proteins and other products that result from the use of DNA technology are found in essentially every western pharmacy, physician or veterinarian office, medical testing laboratory, and biological research laboratory. In addition, organisms that have been manipulated using recombinant DNA technology, as well as products derived from those organisms, have found their way into many farms, supermarkets, home medicine cabinets, and even pet shops, such as those that sell GloFish and other genetically modified animals.

The most common application of recombinant DNA is in basic research, in which the technology is important to most current work in the biological and biomedical sciences. Recombinant DNA is used to identify, map and sequence genes, and to determine their function. rDNA probes are employed in analyzing gene expression within individual cells, and throughout the tissues of whole organisms. Recombinant proteins are widely used as reagents in laboratory experiments and to generate antibody probes for examining protein synthesis within cells and organisms.

Many additional practical applications of recombinant DNA are found in industry, food production, human and veterinary medicine, agriculture, and bioengineering. Some specific examples are identified below.

Recombinant chymosin
Found in rennet, chymosin is an enzyme required to manufacture cheese. It was the first genetically engineered food additive used commercially. Traditionally, processors obtained chymosin from rennet, a preparation derived from the fourth stomach of milk-fed calves. Scientists engineered a non-pathogenic strain (K-12) of E. coli bacteria for large-scale laboratory production of the enzyme. This microbiologically produced recombinant enzyme, identical structurally to the calf derived enzyme, costs less and is produced in abundant quantities. Today about 60\% of U.S. hard cheese is made with genetically engineered chymosin. In 1990, FDA granted chymosin ``generally recognized as safe'' (GRAS) status based on data showing that the enzyme was safe.
Recombinant human insulin
Almost completely replaced insulin obtained from animal sources (e.g.~pigs and cattle) for the treatment of insulin-dependent diabetes. A variety of different recombinant insulin preparations are in widespread use. Recombinant insulin is synthesized by inserting the human insulin gene into E. coli, or yeast (Saccharomyces cerevisiae) which then produces insulin for human use.
Recombinant human growth hormone (HGH, somatotropin)
Administered to patients whose pituitary glands generate insufficient quantities to support normal growth and development. Before recombinant HGH became available, HGH for therapeutic use was obtained from pituitary glands of cadavers. This unsafe practice led to some patients developing Creutzfeldt--Jakob disease. Recombinant HGH eliminated this problem, and is now used therapeutically. It has also been misused as a performance-enhancing drug by athletes and others. DrugBank entry
Recombinant blood clotting factor VIII
A blood-clotting protein that is administered to patients with forms of the bleeding disorder hemophilia, who are unable to produce factor VIII in quantities sufficient to support normal blood coagulation. Before the development of recombinant factor VIII, the protein was obtained by processing large quantities of human blood from multiple donors, which carried a very high risk of transmission of blood borne infectious diseases, for example HIV and hepatitis B. DrugBank entry
Recombinant hepatitis B vaccine
Hepatitis B infection is controlled through the use of a recombinant hepatitis B vaccine, which contains a form of the hepatitis B virus surface antigen that is produced in yeast cells. The development of the recombinant subunit vaccine was an important and necessary development because hepatitis B virus, unlike other common viruses such as polio virus, cannot be grown in vitro. Vaccine information from Hepatitis B Foundation
Diagnosis of infection with HIV
Each of the three widely used methods for diagnosing HIV infection has been developed using recombinant DNA. The antibody test (ELISA or western blot) uses a recombinant HIV protein to test for the presence of antibodies that the body has produced in response to an HIV infection. The DNA test looks for the presence of HIV genetic material using reverse transcription polymerase chain reaction (RT-PCR). Development of the RT-PCR test was made possible by the molecular cloning and sequence analysis of HIV genomes. HIV testing page from US Centers for Disease Control (CDC)
Golden rice
A recombinant variety of rice that has been engineered to express the enzymes responsible for β-carotene biosynthesis. This variety of rice holds substantial promise for reducing the incidence of vitamin A deficiency in the world's population. Golden rice is not currently in use, pending the resolution of regulatory and intellectual property issues.
Herbicide-resistant crops
Commercial varieties of important agricultural crops (including soy, maize/corn, sorghum, canola, alfalfa and cotton) have been developed that incorporate a recombinant gene that results in resistance to the herbicide glyphosate (trade name Roundup), and simplifies weed control by glyphosate application. These crops are in common commercial use in several countries.
Insect-resistant crops
Bacillus thuringeiensis is a bacterium that naturally produces a protein (Bt toxin) with insecticidal properties. The bacterium has been applied to crops as an insect-control strategy for many years, and this practice has been widely adopted in agriculture and gardening. Recently, plants have been developed that express a recombinant form of the bacterial protein, which may effectively control some insect predators. Environmental issues associated with the use of these transgenic crops have not been fully resolved.
History
The idea of recombinant DNA was first proposed by Peter Lobban, a graduate student of Prof.~Dale Kaiser in the Biochemistry Department at Stanford University Medical School. The first publications describing the successful production and intracellular replication of recombinant DNA appeared in 1972 and 1973, from Stanford and UCSF. In 1980 Paul Berg, a professor in the Biochemistry Department at Stanford and an author on one of the first papers was awarded the Nobel Prize in Chemistry for his work on nucleic acids ``with particular regard to recombinant DNA''. Werner Arber, Hamilton Smith, and Daniel Nathans shared the 1978 Nobel Prize in Physiology or Medicine for the discovery of restriction endonucleases which enhanced the techniques of rDNA technology.

Stanford University applied for a US patent on recombinant DNA in 1974, listing the inventors as Herbert W. Boyer (professor at the University of California, San Francisco) and Stanley N. Cohen (professor at Stanford University); this patent was awarded in 1980. The first licensed drug generated using recombinant DNA technology was human insulin, developed by Genentech and licensed by Eli Lilly and Company.

Controversy
Scientists associated with the initial development of recombinant DNA methods recognized that the potential existed for organisms containing recombinant DNA to have undesirable or dangerous properties. At the 1975 Asilomar Conference on Recombinant DNA, these concerns were discussed and a voluntary moratorium on recombinant DNA research was initiated for experiments that were considered particularly risky. This moratorium was widely observed until the National Institutes of Health (USA) developed and issued formal guidelines for rDNA work. Today, recombinant DNA molecules and recombinant proteins are usually not regarded as dangerous. However, concerns remain about some organisms that express recombinant DNA, particularly when they leave the laboratory and are introduced into the environment or food chain. These concerns are discussed in the articles on genetically modified organisms and genetically modified food controversies. Furthermore, there are concerns about the by-products in biopharmaceutical production, where recombinant DNA result in specific protein products. The major by-product, termed host cell protein, comes from the host expression system and poses a threat to the patient's health and the overall environment.

\hypertarget{molecular-cloning}{%
\section{Molecular cloning}\label{molecular-cloning}}

Molecular cloning is a set of experimental methods in molecular biology that are used to assemble recombinant DNA molecules and to direct their replication within host organisms. The use of the word cloning refers to the fact that the method involves the replication of one molecule to produce a population of cells with identical DNA molecules. Molecular cloning generally uses DNA sequences from two different organisms: the species that is the source of the DNA to be cloned, and the species that will serve as the living host for replication of the recombinant DNA. Molecular cloning methods are central to many contemporary areas of modern biology and medicine.

In a conventional molecular cloning experiment, the DNA to be cloned is obtained from an organism of interest, then treated with enzymes in the test tube to generate smaller DNA fragments. Subsequently, these fragments are then combined with vector DNA to generate recombinant DNA molecules. The recombinant DNA is then introduced into a host organism (typically an easy-to-grow, benign, laboratory strain of E. coli bacteria). This will generate a population of organisms in which recombinant DNA molecules are replicated along with the host DNA. Because they contain foreign DNA fragments, these are transgenic or genetically modified microorganisms (GMO). This process takes advantage of the fact that a single bacterial cell can be induced to take up and replicate a single recombinant DNA molecule. This single cell can then be expanded exponentially to generate a large amount of bacteria, each of which contain copies of the original recombinant molecule. Thus, both the resulting bacterial population, and the recombinant DNA molecule, are commonly referred to as ``clones''. Strictly speaking, recombinant DNA refers to DNA molecules, while molecular cloning refers to the experimental methods used to assemble them. The idea arose that different DNA sequences could be inserted into a plasmid and that these foreign sequences would be carried into bacteria and digested as part of the plasmid. That is, these plasmids could serve as cloning vectors to carry genes.

Virtually any DNA sequence can be cloned and amplified, but there are some factors that might limit the success of the process. Examples of the DNA sequences that are difficult to clone are inverted repeats, origins of replication, centromeres and telomeres. Another characteristic that limits chances of success is large size of DNA sequence. Inserts larger than 10kbp have very limited success, but bacteriophages such as bacteriophage λ can be modified to successfully insert a sequence up to 40 kbp.

History
Prior to the 1970s, the understanding of genetics and molecular biology was severely hampered by an inability to isolate and study individual genes from complex organisms. This changed dramatically with the advent of molecular cloning methods. Microbiologists, seeking to understand the molecular mechanisms through which bacteria restricted the growth of bacteriophage, isolated restriction endonucleases, enzymes that could cleave DNA molecules only when specific DNA sequences were encountered. They showed that restriction enzymes cleaved chromosome-length DNA molecules at specific locations, and that specific sections of the larger molecule could be purified by size fractionation. Using a second enzyme, DNA ligase, fragments generated by restriction enzymes could be joined in new combinations, termed recombinant DNA. By recombining DNA segments of interest with vector DNA, such as bacteriophage or plasmids, which naturally replicate inside bacteria, large quantities of purified recombinant DNA molecules could be produced in bacterial cultures. The first recombinant DNA molecules were generated and studied in 1972.

Overview
Molecular cloning takes advantage of the fact that the chemical structure of DNA is fundamentally the same in all living organisms. Therefore, if any segment of DNA from any organism is inserted into a DNA segment containing the molecular sequences required for DNA replication, and the resulting recombinant DNA is introduced into the organism from which the replication sequences were obtained, then the foreign DNA will be replicated along with the host cell's DNA in the transgenic organism.

Molecular cloning is similar to polymerase chain reaction (PCR) in that it permits the replication of DNA sequence. The fundamental difference between the two methods is that molecular cloning involves replication of the DNA in a living microorganism, while PCR replicates DNA in an in vitro solution, free of living cells.

Steps

n standard molecular cloning experiments, the cloning of any DNA fragment essentially involves seven steps: (1) Choice of host organism and cloning vector, (2) Preparation of vector DNA, (3) Preparation of DNA to be cloned, (4) Creation of recombinant DNA, (5) Introduction of recombinant DNA into host organism, (6) Selection of organisms containing recombinant DNA, (7) Screening for clones with desired DNA inserts and biological properties.

Although the detailed planning of the cloning can be done in any text editor, together with online utilities for e.g.~PCR primer design, dedicated software exist for the purpose. Software for the purpose include for example ApE (open source), DNAStrider (open source), Serial Cloner (gratis) and Collagene (open source).

Notably, the growing capacity and fidelity of DNA synthesis platforms allows for increasingly intricate designs in molecular engineering. These projects may include very long strands of novel DNA sequence and/or test entire libraries simultaneously, as opposed to of individual sequences. These shifts introduce complexity that require design to move away from the flat nucleotide-based representation and towards a higher level of abstraction. Examples of such tools are GenoCAD, Teselagen (free for academia) or GeneticConstructor (free for academics).

Choice of host organism and cloning vector

Although a very large number of host organisms and molecular cloning vectors are in use, the great majority of molecular cloning experiments begin with a laboratory strain of the bacterium E. coli (Escherichia coli) and a plasmid cloning vector. E. coli and plasmid vectors are in common use because they are technically sophisticated, versatile, widely available, and offer rapid growth of recombinant organisms with minimal equipment. If the DNA to be cloned is exceptionally large (hundreds of thousands to millions of base pairs), then a bacterial artificial chromosome or yeast artificial chromosome vector is often chosen.

Specialized applications may call for specialized host-vector systems. For example, if the experimentalists wish to harvest a particular protein from the recombinant organism, then an expression vector is chosen that contains appropriate signals for transcription and translation in the desired host organism. Alternatively, if replication of the DNA in different species is desired (for example, transfer of DNA from bacteria to plants), then a multiple host range vector (also termed shuttle vector) may be selected. In practice, however, specialized molecular cloning experiments usually begin with cloning into a bacterial plasmid, followed by subcloning into a specialized vector.

Whatever combination of host and vector are used, the vector almost always contains four DNA segments that are critically important to its function and experimental utility:

\begin{itemize}
\tightlist
\item
  DNA replication origin is necessary for the vector (and its linked recombinant sequences) to replicate inside the host organism
\item
  one or more unique restriction endonuclease recognition sites to serves as sites where foreign DNA may be introduced
\item
  a selectable genetic marker gene that can be used to enable the survival of cells that have taken up vector sequences
\item
  a tag gene that can be used to screen for cells containing the foreign DNA
\end{itemize}

Preparation of vector DNA
The cloning vector is treated with a restriction endonuclease to cleave the DNA at the site where foreign DNA will be inserted. The restriction enzyme is chosen to generate a configuration at the cleavage site that is compatible with the ends of the foreign DNA (see DNA end). Typically, this is done by cleaving the vector DNA and foreign DNA with the same restriction enzyme, for example EcoRI. Most modern vectors contain a variety of convenient cleavage sites that are unique within the vector molecule (so that the vector can only be cleaved at a single site) and are located within a gene (frequently beta-galactosidase) whose inativation can be used to distinguish recombinant from non-recombinant organisms at a later step in the process. To improve the ratio of recombinant to non-recombinant organisms, the cleaved vector may be treated with an enzyme (alkaline phosphatase) that dephosphorylates the vector ends. Vector molecules with dephosphorylated ends are unable to replicate, and replication can only be restored if foreign DNA is integrated into the cleavage site.

Preparation of DNA to be cloned

For cloning of genomic DNA, the DNA to be cloned is extracted from the organism of interest. Virtually any tissue source can be used (even tissues from extinct animals), as long as the DNA is not extensively degraded. The DNA is then purified using simple methods to remove contaminating proteins (extraction with phenol), RNA (ribonuclease) and smaller molecules (precipitation and/or chromatography). Polymerase chain reaction (PCR) methods are often used for amplification of specific DNA or RNA (RT-PCR) sequences prior to molecular cloning.

DNA for cloning experiments may also be obtained from RNA using reverse transcriptase (complementary DNA or cDNA cloning), or in the form of synthetic DNA (artificial gene synthesis). cDNA cloning is usually used to obtain clones representative of the mRNA population of the cells of interest, while synthetic DNA is used to obtain any precise sequence defined by the designer. Such a designed sequence may be required when moving genes across genetic codes (for example, from the mitochrondria to the nucleus) or simply for increasing expression via codon optimization.

The purified DNA is then treated with a restriction enzyme to generate fragments with ends capable of being linked to those of the vector. If necessary, short double-stranded segments of DNA (linkers) containing desired restriction sites may be added to create end structures that are compatible with the vector.

Creation of recombinant DNA with DNA ligase
The creation of recombinant DNA is in many ways the simplest step of the molecular cloning process. DNA prepared from the vector and foreign source are simply mixed together at appropriate concentrations and exposed to an enzyme (DNA ligase) that covalently links the ends together. This joining reaction is often termed ligation. The resulting DNA mixture containing randomly joined ends is then ready for introduction into the host organism.

DNA ligase only recognizes and acts on the ends of linear DNA molecules, usually resulting in a complex mixture of DNA molecules with randomly joined ends. The desired products (vector DNA covalently linked to foreign DNA) will be present, but other sequences (e.g.~foreign DNA linked to itself, vector DNA linked to itself and higher-order combinations of vector and foreign DNA) are also usually present. This complex mixture is sorted out in subsequent steps of the cloning process, after the DNA mixture is introduced into cells.

Introduction of recombinant DNA into host organism
The DNA mixture, previously manipulated in vitro, is moved back into a living cell, referred to as the host organism. The methods used to get DNA into cells are varied, and the name applied to this step in the molecular cloning process will often depend upon the experimental method that is chosen (e.g.~transformation, transduction, transfection, electroporation).

When microorganisms are able to take up and replicate DNA from their local environment, the process is termed transformation, and cells that are in a physiological state such that they can take up DNA are said to be competent. In mammalian cell culture, the analogous process of introducing DNA into cells is commonly termed transfection. Both transformation and transfection usually require preparation of the cells through a special growth regime and chemical treatment process that will vary with the specific species and cell types that are used.

Electroporation uses high voltage electrical pulses to translocate DNA across the cell membrane (and cell wall, if present). In contrast, transduction involves the packaging of DNA into virus-derived particles, and using these virus-like particles to introduce the encapsulated DNA into the cell through a process resembling viral infection. Although electroporation and transduction are highly specialized methods, they may be the most efficient methods to move DNA into cells.

Selection of organisms containing vector sequences
Whichever method is used, the introduction of recombinant DNA into the chosen host organism is usually a low efficiency process; that is, only a small fraction of the cells will actually take up DNA. Experimental scientists deal with this issue through a step of artificial genetic selection, in which cells that have not taken up DNA are selectively killed, and only those cells that can actively replicate DNA containing the selectable marker gene encoded by the vector are able to survive.

When bacterial cells are used as host organisms, the selectable marker is usually a gene that confers resistance to an antibiotic that would otherwise kill the cells, typically ampicillin. Cells harboring the plasmid will survive when exposed to the antibiotic, while those that have failed to take up plasmid sequences will die. When mammalian cells (e.g.~human or mouse cells) are used, a similar strategy is used, except that the marker gene (in this case typically encoded as part of the kanMX cassette) confers resistance to the antibiotic Geneticin.

Screening for clones with desired DNA inserts and biological properties
Modern bacterial cloning vectors (e.g.~pUC19 and later derivatives including the pGEM vectors) use the blue-white screening system to distinguish colonies (clones) of transgenic cells from those that contain the parental vector (i.e.~vector DNA with no recombinant sequence inserted). In these vectors, foreign DNA is inserted into a sequence that encodes an essential part of beta-galactosidase, an enzyme whose activity results in formation of a blue-colored colony on the culture medium that is used for this work. Insertion of the foreign DNA into the beta-galactosidase coding sequence disables the function of the enzyme, so that colonies containing transformed DNA remain colorless (white). Therefore, experimentalists are easily able to identify and conduct further studies on transgenic bacterial clones, while ignoring those that do not contain recombinant DNA.

The total population of individual clones obtained in a molecular cloning experiment is often termed a DNA library. Libraries may be highly complex (as when cloning complete genomic DNA from an organism) or relatively simple (as when moving a previously cloned DNA fragment into a different plasmid), but it is almost always necessary to examine a number of different clones to be sure that the desired DNA construct is obtained. This may be accomplished through a very wide range of experimental methods, including the use of nucleic acid hybridizations, antibody probes, polymerase chain reaction, restriction fragment analysis and/or DNA sequencing.

Applications
Molecular cloning provides scientists with an essentially unlimited quantity of any individual DNA segments derived from any genome. This material can be used for a wide range of purposes, including those in both basic and applied biological science. A few of the more important applications are summarized here.

Genome organization and gene expression
Molecular cloning has led directly to the elucidation of the complete DNA sequence of the genomes of a very large number of species and to an exploration of genetic diversity within individual species, work that has been done mostly by determining the DNA sequence of large numbers of randomly cloned fragments of the genome, and assembling the overlapping sequences.

At the level of individual genes, molecular clones are used to generate probes that are used for examining how genes are expressed, and how that expression is related to other processes in biology, including the metabolic environment, extracellular signals, development, learning, senescence and cell death. Cloned genes can also provide tools to examine the biological function and importance of individual genes, by allowing investigators to inactivate the genes, or make more subtle mutations using regional mutagenesis or site-directed mutagenesis. Genes cloned into expression vectors for functional cloning provide a means to screen for genes on the basis of the expressed protein's function.

Production of recombinant proteins
Obtaining the molecular clone of a gene can lead to the development of organisms that produce the protein product of the cloned genes, termed a recombinant protein. In practice, it is frequently more difficult to develop an organism that produces an active form of the recombinant protein in desirable quantities than it is to clone the gene. This is because the molecular signals for gene expression are complex and variable, and because protein folding, stability and transport can be very challenging.

Many useful proteins are currently available as recombinant products. These include--(1) medically useful proteins whose administration can correct a defective or poorly expressed gene (e.g.~recombinant factor VIII, a blood-clotting factor deficient in some forms of hemophilia, and recombinant insulin, used to treat some forms of diabetes), (2) proteins that can be administered to assist in a life-threatening emergency (e.g.~tissue plasminogen activator, used to treat strokes), (3) recombinant subunit vaccines, in which a purified protein can be used to immunize patients against infectious diseases, without exposing them to the infectious agent itself (e.g.~hepatitis B vaccine), and (4) recombinant proteins as standard material for diagnostic laboratory tests.

Transgenic organisms
Once characterized and manipulated to provide signals for appropriate expression, cloned genes may be inserted into organisms, generating transgenic organisms, also termed genetically modified organisms (GMOs). Although most GMOs are generated for purposes of basic biological research (see for example, transgenic mouse), a number of GMOs have been developed for commercial use, ranging from animals and plants that produce pharmaceuticals or other compounds (pharming), herbicide-resistant crop plants, and fluorescent tropical fish (GloFish) for home entertainment.

Gene therapy
Gene therapy involves supplying a functional gene to cells lacking that function, with the aim of correcting a genetic disorder or acquired disease. Gene therapy can be broadly divided into two categories. The first is alteration of germ cells, that is, sperm or eggs, which results in a permanent genetic change for the whole organism and subsequent generations. This ``germ line gene therapy'' is considered by many to be unethical in human beings. The second type of gene therapy, ``somatic cell gene therapy'', is analogous to an organ transplant. In this case, one or more specific tissues are targeted by direct treatment or by removal of the tissue, addition of the therapeutic gene or genes in the laboratory, and return of the treated cells to the patient. Clinical trials of somatic cell gene therapy began in the late 1990s, mostly for the treatment of cancers and blood, liver, and lung disorders.

Despite a great deal of publicity and promises, the history of human gene therapy has been characterized by relatively limited success. The effect of introducing a gene into cells often promotes only partial and/or transient relief from the symptoms of the disease being treated. Some gene therapy trial patients have suffered adverse consequences of the treatment itself, including deaths. In some cases, the adverse effects result from disruption of essential genes within the patient's genome by insertional inactivation. In others, viral vectors used for gene therapy have been contaminated with infectious virus. Nevertheless, gene therapy is still held to be a promising future area of medicine, and is an area where there is a significant level of research and development activity.

\hypertarget{mutagenesis-1}{%
\section{Mutagenesis}\label{mutagenesis-1}}

In molecular biology, mutagenesis is an important laboratory technique whereby DNA mutations are deliberately engineered to produce mutant genes, proteins, strains of bacteria, or other genetically modified organisms. The various constituents of a gene, as well as its regulatory elements and its gene products, may be mutated so that the functioning of a genetic locus, process, or product can be examined in detail. The mutation may produce mutant proteins with interesting properties or enhanced or novel functions that may be of commercial use. Mutant strains may also be produced that have practical application or allow the molecular basis of a particular cell function to be investigated.

A large number of methods for achieving experimental mutagenesis have been developed. Initially, the kind of mutations artificially induced in the laboratory were entirely random; methods allowing for more specific site-directed mutagenesis were introduced later. Since 2013, development of the CRISPR/Cas9 technology, based on a prokaryotic viral defense system, has allowed for the editing or mutagenesis of a genome in vivo.

Random mutagenesis
Early approaches to mutagenesis relied on methods which produced entirely random mutations. In such methods, cells or organisms are exposed to mutagens such as UV radiation or mutagenic chemicals, and mutants with desired characteristics are then selected. Hermann Muller discovered in 1927 that X-rays can cause genetic mutations in fruit flies, and went on to use the mutants he created for his studies in genetics. For Escherichia coli, mutants may be selected first by exposure to UV radiation, then plated onto an agar medium. The colonies formed are then replica-plated, one in a rich medium, another in a minimal medium, and mutants that have specific nutritional requirements can then be identified by their inability to grow in the minimal medium. Similar procedures may be repeated with other types of cells and with different media for selection.

A number of methods for generating random mutations in specific proteins were later developed to screen for mutants with interesting or improved properties. These methods may involve the use of doped nucleotides in oligonucleotide synthesis, or conducting a PCR reaction in conditions that enhance misincorporation of nucleotides (error-prone PCR), for example by reducing the fidelity of replication or using nucleotide analogues. A variation of this method for integrating non-biased mutations in a gene is sequence saturation mutagenesis. PCR products which contain mutation(s) are then cloned into an expression vector and the mutant proteins produced can then be characterised.

In animal studies, alkylating agents such as N-ethyl-N-nitrosourea (ENU) have been used to generate mutant mice. Ethyl methanesulfonate (EMS) is also often used to generate animal and plant mutants.

In a European Union law (as 2001/18 directive), this kind of mutagenesis may be used to produce GMOs but the products are exempted from regulation: no labeling, no evaluation.

Site-directed mutagenesis
Main article: Site-directed mutagenesis
Many researchers seek to introduce selected changes to DNA in a precise, site-specific manner. Analogs of nucleotides and other chemicals were first used to generate localized point mutations. Such chemicals include aminopurine, which induces an AT to GC transition, while nitrosoguanidine, bisulfite, and N4-hydroxycytidine may induce a GC to AT transition. These techniques allow specific mutations to be engineered into a protein; however, they are not flexible with respect to the kinds of mutants generated, nor are they as specific as later methods of site-directed mutagenesis and therefore have some degree of randomness.

Current techniques for site-specific mutation commonly involve using pre-fabricated mutagenic oligonucleotides in a primer extension reaction with DNA polymerase. This methods allows for point mutation or deletion or insertion of small stretches of DNA at specific sites. Advances in methodology have made such mutagenesis now a relatively simple and efficient process.

The site-directed approach may be done systematically in such techniques as alanine scanning mutagenesis, whereby residues are systematically mutated to alanine in order to identify residues important to the structure or function of a protein.

Combinatorial mutagenesis
Combinatorial mutagenesis is a technique whereby large number of mutants may be screened for a particular characteristic. In this technique, a few selected positions or a short stretch of DNA may be exhaustively modified to obtain a comprehensive library of mutant proteins. One approach of this technique is to excise a portion of DNA and replaced with a library of sequences containing all possible combinations at the desired mutation sites. The segment may be at an enzyme active site, or sequences that have structural significance or immunogenic property. A segment however may also be inserted randomly into the gene in order to assess the structural or functional significance of particular part of protein.

Insertional mutagenesis
Further information: Signature tagged mutagenesis and Transposon mutagenesis
In cancer research engineered mutations also provide mechanistic insights into the development of the disease. Insertional mutagenesis using transposons, retrovirus such as mouse mammary tumor virus and murine leukemia virus may be used to identify genes involved in carcinogenesis and to understand the biological pathways of specific cancer. Various insertional mutagenesis techniques may also be used to study the function of particular gene.

Homologous recombination
Homologous recombination can be used to produce specific mutation in an organism. Vector containing DNA sequence similar to the gene to be modified is introduced to the cell, and by a process of recombination replaces the target gene in the chromosome. This method can be used to introduce a mutation or knock out a gene, for example as used in the production of knockout mice.

Gene synthesis
As the cost of DNA oligonucleotide synthesis falls, artificial synthesis of a complete gene is now a viable method for introducing mutations into a gene. This method allows for extensive mutation at multiple sites, including the complete redesign of the codon usage of a gene to optimise it for a particular organism.

\hypertarget{gene-knockout}{%
\section{Gene knockout}\label{gene-knockout}}

A gene knockout (abbreviation: KO) is a genetic technique in which one of an organism's genes is made inoperative (``knocked out'' of the organism). However, KO can also refer to the gene that is knocked out or the organism that carries the gene knockout. Knockout organisms or simply knockouts are used to study gene function, usually by investigating the effect of gene loss. Researchers draw inferences from the difference between the knockout organism and normal individuals.

The KO technique is essentially the opposite of a gene knock-in. Knocking out two genes simultaneously in an organism is known as a double knockout (DKO). Similarly the terms triple knockout (TKO) and quadruple knockouts (QKO) are used to describe three or four knocked out genes, respectively. However, one needs to distinguish between heterozygous and homozygous KOs. In the former, only one of two gene copies (alleles) is knocked out, in the latter both are knocked out.

Methods
Knockouts are accomplished through a variety of techniques. Originally, naturally occurring mutations were identified and then gene loss or inactivation had to be established by DNA sequencing or other methods.

Homologous recombination

Traditionally, homologous recombination was the main method for causing a gene knockout. This method involves creating a DNA construct containing the desired mutation. For knockout purposes, this typically involves a drug resistance marker in place of the desired knockout gene. The construct will also contain a minimum of 2kb of homology to the target sequence. The construct can be delivered to stem cells either through microinjection or electroporation. This method then relies on the cell's own repair mechanisms to recombine the DNA construct into the existing DNA. This results in the sequence of the gene being altered, and most cases the gene will b translated into a nonfunctional protein, if it is translated at all. However, this is an inefficient process, as homologous recombination accounts for only 10−2 to 10-3 of DNA integrations. Often, the drug selection marker on the construct is used to select for cells in which the recombination event has occurred.e

These stem cells now lacking the gene could be used in vivo, for instance in mice, by inserting them into early embryos. If the resulting chimeric mouse contained the genetic change in their germline, this could then be passed on offspring.

In diploid organisms, which contain two alleles for most genes, and may as well contain several related genes that collaborate in the same role, additional rounds of transformation and selection are performed until every targeted gene is knocked out. Selective breeding may be required to produce homozygous knockout animals.

Site-specific nucleases

There are currently three methods in use that involve precisely targeting a DNA sequence in order to introduce a double-stranded break. Once this occurs, the cell's repair mechanisms will attempt to repair this double stranded break, often through non-homologous end joining (NHEJ), which involves directly ligating the two cut ends together. This may be done imperfectly, therefore sometimes causing insertions or deletions of base pairs, which cause frameshift mutations. These mutations can render the gene in which they occur nonfunctional, thus creating a knockout of that gene. This process is more efficient than homologous recombination, and therefore can be more easily used to create biallelic knockouts.

Zinc-fingers
Main article: Zinc finger nuclease
Zinc-finger nucleases consist of DNA binding domains that can precisely target a DNA sequence. Each zinc finger can recognize codons of a desired DNA sequence, and therefore can be modularly assembled to bind to a particular sequence. These binding domains are coupled with a restriction endonuclease that can cause a double stranded break (DSB) in the DNA. Repair processes may introduce mutations that destroy functionality of the gene.

TALENS

Transcription activator-like effector nucleases (TALENs) also contain a DNA binding domain and a nuclease that can cleave DNA. The DNA binding region consists of amino acid repeats that each recognize a single base pair of the desired targeted DNA sequence. If this cleavage is targeted to a gene coding region, and NHEJ-mediated repair introduces insertions and deletions, a frameshift mutation often results, thus disrupting function of the gene.

CRISPR/Cas9
Clustered regularly interspaced short palindromic repeats (CRISPR)/Cas9 is a method for genome editing that contains a guide RNA complexed with a Cas9 protein. The guide RNA can be engineered to match a desired DNA sequence through simple complementary base pairing, as opposed to the time consuming assembly of constructs required by zinc-fingers or TALENs. The coupled Cas9 will cause a double stranded break in the DNA. Following the same principle as zinc-fingers and TALENs, the attempts to repair these double stranded breaks often result in frameshift mutations that result in an nonfunctional gene.

Knockin
Gene knockin is similar to gene knockout, but it replaces a gene with another instead of deleting it.

Types
Conditional knockouts
A conditional knockout allows gene deletion in a tissue in a time specific manner. This is required in place of a gene knockout if the null mutation would lead to embryonic death. This is done by introducing short sequences called loxP sites around the gene. These sequences will be introduced into the germ-line via the same mechanism as a knock-out. This germ-line can then be crossed to another germline containing Cre-recombinase which is a viral enzyme that can recognize these sequences, recombines them and deletes the gene flanked by these sites.

Use

Knockouts are primarily used to understand the role of a specific gene or DNA region by comparing the knockout organism to a wildtype with a similar genetic background.

Knockout organisms are also used as screening tools in the development of drugs, to target specific biological processes or deficiencies by using a specific knockout, or to understand the mechanism of action of a drug by using a library of knockout organisms spanning the entire genome, such as in Saccharomyces cerevisiae.

Conditional gene knockout

Conditional gene knockout is a technique used to eliminate a specific gene in a certain tissue, such as the liver. This technique is useful to study the role of individual genes in living organisms. It differs from traditional gene knockout because it targets specific genes at specific times rather than being deleted from beginning of life. Using the conditional gene knockout technique eliminates many of the side effects from traditional gene knockout. In traditional gene knockout, embryonic death from a gene mutation can occur, and this prevents scientists from studying the gene in adults. Some tissues cannot be studied properly in isolation, so the gene must be inactive in a certain tissue while remaining active in others. With this technology, scientists are able to knockout genes at a specific stage in development and study how the knockout of a gene in one tissue affects the same gene in other tissues.

Technique

The most commonly used technique is the Cre-lox recombination system. The Cre recombinase enzyme specifically recognizes two lox (loci of recombination) sites within DNA and causes recombination between them. During recombination two strands of DNA exchange information. This recombination will cause a deletion or inversion of the genes between the two lox sites, depending on their orientation. An entire gene can be removed to inactivate it. This whole system is inducible so a chemical can be added to knock genes out at a specific time. Two of the most commonly used chemicals are tetracycline, which activates transcription of the Cre recombinase gene and tamoxifen, which activates transport of the Cre recombinase protein to the nucleus. Only a few cell types express Cre recombinase and no mammalian cells express it so there is no risk of accidental activation of lox sites when using conditional gene knockout in mammals. Figuring out how to express Cre-recombinase in an organism tends to be the most difficult part of this technique.
Uses
The conditional gene knockout method is often used to model human diseases in other mammals. It has increased scientists' ability to study diseases, such as cancer, that develop in specific cell types or developmental stages. It is known that mutations in the BRCA1 gene are linked to breast cancer. Scientists used conditional gene knockout to delete the BRCA1 allele in mammary gland tissue in mice and found that it plays an important role in tumour suppression.

A specific gene in mouse brain thought to be involved in the onset of Alzheimer's disease which codes for the enzyme cyclin-dependent kinase 5 (Cdk5) was knocked out. Such mice were found to be `smarter' than normal mice and were able to handle complex tasks more intelligently compared to `normal' mice bred in the laboratory.

Knockout Mouse Project (KOMP)
Conditional gene knockouts in mice are often used to study human diseases because many genes produce similar phenotypes in both species. The goal of KOMP is to create knockout mutations in the embryonic stem cells for each of the 20,000 protein coding genes in mice. The genes are knocked out because this is the best way to study their function and learn more about their role in human diseases. Some alleles in this project cannot be knocked out using traditional methods and require the specificity of the conditional gene knockout technique. Other combinatorial methods are needed to knockout the last remaining alleles. Conditional gene knockout is a time-consuming procedure and there are additional projects focusing on knocking out the remaining mouse genes. The KOMP projected was started in 2006 and is still ongoing.

\hypertarget{rna-interference}{%
\section{RNA interference}\label{rna-interference}}

RNA interference (RNAi) is a biological process in which RNA molecules inhibit gene expression or translation, by neutralizing targeted mRNA molecules. The process of RNAi was referred to as ``co-suppression'' and ``quelling'' when observed prior to the knowledge of an RNA-related mechanism. The discovery of RNAi was preceded first by observations of transcriptional inhibition by antisense RNA expressed in transgenic plants, and more directly by reports of unexpected outcomes in experiments performed by plant scientists in the United States and the Netherlands in the early 1990s. The detailed study of each of these seemingly different processes elucidated that the identity of these phenomena were all actually RNAi. Andrew Fire and Craig C. Mello shared the 2006 Nobel Prize in Physiology or Medicine for their work on RNA interference in the nematode worm Caenorhabditis elegans, which they published in 1998. Since the discovery of RNAi and its regulatory potentials, it has become evident that RNAi has immense potential in suppression of desired genes. RNAi is now known as precise, efficient, stable and better than antisense technology for gene suppression. However, antisense RNA produced intracellularly by an expression vector may be developed and find utility as novel therapeutic agents.

Two types of small ribonucleic acid (RNA) molecules -- microRNA (miRNA) and small interfering RNA (siRNA) -- are central to RNA interference. RNAs are the direct products of genes, and these small RNAs can direct enzyme complexes to degrade messenger RNA (mRNA) molecules and thus decrease their activity by preventing translation, via post-transcriptional gene silencing. Moreover, transcription can be inhibited via the pre-transcriptional silencing mechanism of RNA interference, through which an enzyme complex catalyzes DNA methylation at genomic positions complementary to complexed siRNA or miRNA. RNA interference has an important role in defending cells against parasitic nucleotide sequences -- viruses and transposons. It also influences development.

The RNAi pathway is found in many eukaryotes, including animals, and is initiated by the enzyme Dicer, which cleaves long double-stranded RNA (dsRNA) molecules into short double-stranded fragments of \textasciitilde{}21 nucleotide siRNAs. Each siRNA is unwound into two single-stranded RNAs (ssRNAs), the passenger strand and the guide strand. The passenger strand is degraded and the guide strand is incorporated into the RNA-induced silencing complex (RISC). The most well-studied outcome is post-transcriptional gene silencing, which occurs when the guide strand pairs with a complementary sequence in a messenger RNA molecule and induces cleavage by Argonaute 2 (Ago2), the catalytic component of the RISC. In some organisms, this process spreads systemically, despite the initially limited molar concentrations of siRNA.

RNAi is a valuable research tool, both in cell culture and in living organisms, because synthetic dsRNA introduced into cells can selectively and robustly induce suppression of specific genes of interest. RNAi may be used for large-scale screens that systematically shut down each gene in the cell, which can help to identify the components necessary for a particular cellular process or an event such as cell division. The pathway is also used as a practical tool in biotechnology, medicine and insecticides.

RNAi is an RNA-dependent gene silencing process that is controlled by the RNA-induced silencing complex (RISC) and is initiated by short double-stranded RNA molecules in a cell's cytoplasm, where they interact with the catalytic RISC component argonaute. When the dsRNA is exogenous (coming from infection by a virus with an RNA genome or laboratory manipulations), the RNA is imported directly into the cytoplasm and cleaved to short fragments by Dicer. The initiating dsRNA can also be endogenous (originating in the cell), as in pre-microRNAs expressed from RNA-coding genes in the genome. The primary transcripts from such genes are first processed to form the characteristic stem-loop structure of pre-miRNA in the nucleus, then exported to the cytoplasm. Thus, the two dsRNA pathways, exogenous and endogenous, converge at the RISC.

Exogenous dsRNA initiates RNAi by activating the ribonuclease protein Dicer, which binds and cleaves double-stranded RNAs (dsRNAs) in plants, or short hairpin RNAs (shRNAs) in humans, to produce double-stranded fragments of 20--25 base pairs with a 2-nucleotide overhang at the 3' end. Bioinformatics studies on the genomes of multiple organisms suggest this length maximizes target-gene specificity and minimizes non-specific effects. These short double-stranded fragments are called small interfering RNAs (siRNAs). These siRNAs are then separated into single strands and integrated into an active RISC, by RISC-Loading Complex (RLC). RLC includes Dicer-2 and R2D2, and is crucial to unite Ago2 and RISC. TATA-binding protein-associated factor 11 (TAF11) assembles the RLC by facilitating Dcr-2-R2D2 tetramerization, which increases the binding affinity to siRNA by 10-fold. Association with TAF11 would convert the R2-D2-Initiator (RDI) complex into the RLC. R2D2 carries tandem double-stranded RNA-binding domains to recognize the thermodynamically stable terminus of siRNA duplexes, whereas Dicer-2 the other less stable extremity. Loading is asymmetric: the MID domain of Ago2 recognizes the thermodynamically stable end of the siRNA. Therefore, the ``passenger'' (sense) strand whose 5′ end is discarded by MID is ejected, while the saved ``guide'' (antisense) strand cooperates with AGO to form the RISC.

After integration into the RISC, siRNAs base-pair to their target mRNA and cleave it, thereby preventing it from being used as a translation template. Differently from siRNA, a miRNA-loaded RISC complex scans cytoplasmic mRNAs for potential complementarity. Instead of destructive cleavage (by Ago2), miRNAs rather target the 3′ untranslated region (UTR) regions of mRNAs where they typically bind with imperfect complementarity, thus blocking the access of ribosomes for translation.

Exogenous dsRNA is detected and bound by an effector protein, known as RDE-4 in C. elegans and R2D2 in Drosophila, that stimulates dicer activity. The mechanism producing this length specificity is unknown and this protein only binds long dsRNAs.

In C. elegans this initiation response is amplified through the synthesis of a population of `secondary' siRNAs during which the dicer-produced initiating or `primary' siRNAs are used as templates. These `secondary' siRNAs are structurally distinct from dicer-produced siRNAs and appear to be produced by an RNA-dependent RNA polymerase (RdRP).

MicroRNA

MicroRNAs (miRNAs) are genomically encoded non-coding RNAs that help regulate gene expression, particularly during development. The phenomenon of RNA interference, broadly defined, includes the endogenously induced gene silencing effects of miRNAs as well as silencing triggered by foreign dsRNA. Mature miRNAs are structurally similar to siRNAs produced from exogenous dsRNA, but before reaching maturity, miRNAs must first undergo extensive post-transcriptional modification. A miRNA is expressed from a much longer RNA-coding gene as a primary transcript known as a pri-miRNA which is processed, in the cell nucleus, to a 70-nucleotide stem-loop structure called a pre-miRNA by the microprocessor complex. This complex consists of an RNase III enzyme called Drosha and a dsRNA-binding protein DGCR8. The dsRNA portion of this pre-miRNA is bound and cleaved by Dicer to produce the mature miRNA molecule that can be integrated into the RISC complex; thus, miRNA and siRNA share the same downstream cellular machinery. First, viral encoded miRNA was described in EBV. Thereafter, an increasing number of microRNAs have been described in viruses. VIRmiRNA is a comprehensive catalogue covering viral microRNA, their targets and anti-viral miRNAs (see also VIRmiRNA resource: \url{http://crdd.osdd.net/servers/virmirna/}).

siRNAs derived from long dsRNA precursors differ from miRNAs in that miRNAs, especially those in animals, typically have incomplete base pairing to a target and inhibit the translation of many different mRNAs with similar sequences. In contrast, siRNAs typically base-pair perfectly and induce mRNA cleavage only in a single, specific target. In Drosophila and C. elegans, miRNA and siRNA are processed by distinct argonaute proteins and dicer enzymes.

Three prime untranslated regions and microRNAs
Main article: Three prime untranslated region
Three prime untranslated regions (3'UTRs) of messenger RNAs (mRNAs) often contain regulatory sequences that post-transcriptionally cause RNA interference. Such 3'-UTRs often contain both binding sites for microRNAs (miRNAs) as well as for regulatory proteins. By binding to specific sites within the 3'-UTR, miRNAs can decrease gene expression of various mRNAs by either inhibiting translation or directly causing degradation of the transcript. The 3'-UTR also may have silencer regions that bind repressor proteins that inhibit the expression of a mRNA.

The 3'-UTR often contains microRNA response elements (MREs). MREs are sequences to which miRNAs bind. These are prevalent motifs within 3'-UTRs. Among all regulatory motifs within the 3'-UTRs (e.g.~including silencer regions), MREs make up about half of the motifs.

As of 2014, the miRBase web site, an archive of miRNA sequences and annotations, listed 28,645 entries in 233 biologic species. Of these, 1,881 miRNAs were in annotated human miRNA loci. miRNAs were predicted to have an average of about four hundred target mRNAs (affecting expression of several hundred genes). Friedman et al.~estimate that \textgreater{}45,000 miRNA target sites within human mRNA 3'UTRs are conserved above background levels, and \textgreater{}60\% of human protein-coding genes have been under selective pressure to maintain pairing to miRNAs.

Direct experiments show that a single miRNA can reduce the stability of hundreds of unique mRNAs. Other experiments show that a single miRNA may repress the production of hundreds of proteins, but that this repression often is relatively mild (less than 2-fold).

The effects of miRNA dysregulation of gene expression seem to be important in cancer. For instance, in gastrointestinal cancers, nine miRNAs have been identified as epigenetically altered and effective in down regulating DNA repair enzymes.

The effects of miRNA dysregulation of gene expression also seem to be important in neuropsychiatric disorders, such as schizophrenia, bipolar disorder, major depression, Parkinson's disease, Alzheimer's disease and autism spectrum disorders.

RISC activation and catalysis
Exogenous dsRNA is detected and bound by an effector protein, known as RDE-4 in C. elegans and R2D2 in Drosophila, that stimulates dicer activity. This protein only binds long dsRNAs, but the mechanism producing this length specificity is unknown. This RNA-binding protein then facilitates the transfer of cleaved siRNAs to the RISC complex.

In C. elegans this initiation response is amplified through the synthesis of a population of `secondary' siRNAs during which the dicer-produced initiating or `primary' siRNAs are used as templates. These `secondary' siRNAs are structurally distinct from dicer-produced siRNAs and appear to be produced by an RNA-dependent RNA polymerase (RdRP).

The active components of an RNA-induced silencing complex (RISC) are endonucleases called argonaute proteins, which cleave the target mRNA strand complementary to their bound siRNA. As the fragments produced by dicer are double-stranded, they could each in theory produce a functional siRNA. However, only one of the two strands, which is known as the guide strand, binds the argonaute protein and directs gene silencing. The other anti-guide strand or passenger strand is degraded during RISC activation. Although it was first believed that an ATP-dependent helicase separated these two strands, the process proved to be ATP-independent and performed directly by the protein components of RISC. However, an in vitro kinetic analysis of RNAi in the presence and absence of ATP showed that ATP may be required to unwind and remove the cleaved mRNA strand from the RISC complex after catalysis. The guide strand tends to be the one whose 5' end is less stably paired to its complement, but strand selection is unaffected by the direction in which dicer cleaves the dsRNA before RISC incorporation. Instead, the R2D2 protein may serve as the differentiating factor by binding the more-stable 5' end of the passenger strand.

The structural basis for binding of RNA to the argonaute protein was examined by X-ray crystallography of the binding domain of an RNA-bound argonaute protein. Here, the phosphorylated 5' end of the RNA strand enters a conserved basic surface pocket and makes contacts through a divalent cation (an atom with two positive charges) such as magnesium and by aromatic stacking (a process that allows more than one atom to share an electron by passing it back and forth) between the 5' nucleotide in the siRNA and a conserved tyrosine residue. This site is thought to form a nucleation site for the binding of the siRNA to its mRNA target. Analysis of the inhibitory effect of mismatches in either the 5' or 3' end of the guide strand has demonstrated that the 5' end of the guide strand is likely responsible for matching and binding the target mRNA, while the 3' end is responsible for physically arranging target mRNA into a cleavage-favorable RISC region.

It is not understood how the activated RISC complex locates complementary mRNAs within the cell. Although the cleavage process has been proposed to be linked to translation, translation of the mRNA target is not essential for RNAi-mediated degradation. Indeed, RNAi may be more effective against mRNA targets that are not translated. Argonaute proteins are localized to specific regions in the cytoplasm called P-bodies (also cytoplasmic bodies or GW bodies), which are regions with high rates of mRNA decay; miRNA activity is also clustered in P-bodies. Disruption of P-bodies decreases the efficiency of RNA interference, suggesting that they are a critical site in the RNAi process.

Transcriptional silencing

Components of the RNAi pathway are used in many eukaryotes in the maintenance of the organization and structure of their genomes. Modification of histones and associated induction of heterochromatin formation serves to downregulate genes pre-transcriptionally; this process is referred to as RNA-induced transcriptional silencing (RITS), and is carried out by a complex of proteins called the RITS complex. In fission yeast this complex contains argonaute, a chromodomain protein Chp1, and a protein called Tas3 of unknown function. As a consequence, the induction and spread of heterochromatic regions requires the argonaute and RdRP proteins. Indeed, deletion of these genes in the fission yeast S. pombe disrupts histone methylation and centromere formation, causing slow or stalled anaphase during cell division. In some cases, similar processes associated with histone modification have been observed to transcriptionally upregulate genes.

The mechanism by which the RITS complex induces heterochromatin formation and organization is not well understood. Most studies have focused on the mating-type region in fission yeast, which may not be representative of activities in other genomic regions/organisms. In maintenance of existing heterochromatin regions, RITS forms a complex with siRNAs complementary to the local genes and stably binds local methylated histones, acting co-transcriptionally to degrade any nascent pre-mRNA transcripts that are initiated by RNA polymerase. The formation of such a heterochromatin region, though not its maintenance, is dicer-dependent, presumably because dicer is required to generate the initial complement of siRNAs that target subsequent transcripts. Heterochromatin maintenance has been suggested to function as a self-reinforcing feedback loop, as new siRNAs are formed from the occasional nascent transcripts by RdRP for incorporation into local RITS complexes. The relevance of observations from fission yeast mating-type regions and centromeres to mammals is not clear, as heterochromatin maintenance in mammalian cells may be independent of the components of the RNAi pathway.

Crosstalk with RNA editing
The type of RNA editing that is most prevalent in higher eukaryotes converts adenosine nucleotides into inosine in dsRNAs via the enzyme adenosine deaminase (ADAR). It was originally proposed in 2000 that the RNAi and A→I RNA editing pathways might compete for a common dsRNA substrate. Some pre-miRNAs do undergo A→I RNA editing and this mechanism may regulate the processing and expression of mature miRNAs. Furthermore, at least one mammalian ADAR can sequester siRNAs from RNAi pathway components. Further support for this model comes from studies on ADAR-null C. elegans strains indicating that A→I RNA editing may counteract RNAi silencing of endogenous genes and transgenes.

Gene knockdown
The RNA interference pathway is often exploited in experimental biology to study the function of genes in cell culture and in vivo in model organisms. Double-stranded RNA is synthesized with a sequence complementary to a gene of interest and introduced into a cell or organism, where it is recognized as exogenous genetic material and activates the RNAi pathway. Using this mechanism, researchers can cause a drastic decrease in the expression of a targeted gene. Studying the effects of this decrease can show the physiological role of the gene product. Since RNAi may not totally abolish expression of the gene, this technique is sometimes referred as a ``knockdown'', to distinguish it from ``knockout'' procedures in which expression of a gene is entirely eliminated. In a recent study validation of RNAi silencing efficiency using gene array data showed 18.5\% failure rate across 429 independent experiments.

Extensive efforts in computational biology have been directed toward the design of successful dsRNA reagents that maximize gene knockdown but minimize ``off-target'' effects. Off-target effects arise when an introduced RNA has a base sequence that can pair with and thus reduce the expression of multiple genes. Such problems occur more frequently when the dsRNA contains repetitive sequences. It has been estimated from studying the genomes of humans, C. elegans and S. pombe that about 10\% of possible siRNAs have substantial off-target effects. A multitude of software tools have been developed implementing algorithms for the design of general mammal-specific, and virus-specific siRNAs that are automatically checked for possible cross-reactivity.

Depending on the organism and experimental system, the exogenous RNA may be a long strand designed to be cleaved by dicer, or short RNAs designed to serve as siRNA substrates. In most mammalian cells, shorter RNAs are used because long double-stranded RNA molecules induce the mammalian interferon response, a form of innate immunity that reacts nonspecifically to foreign genetic material. Mouse oocytes and cells from early mouse embryos lack this reaction to exogenous dsRNA and are therefore a common model system for studying mammalian gene-knockdown effects. Specialized laboratory techniques have also been developed to improve the utility of RNAi in mammalian systems by avoiding the direct introduction of siRNA, for example, by stable transfection with a plasmid encoding the appropriate sequence from which siRNAs can be transcribed, or by more elaborate lentiviral vector systems allowing the inducible activation or deactivation of transcription, known as conditional RNAi.

Most functional genomics applications of RNAi in animals have used C. elegans and Drosophila, as these are the common model organisms in which RNAi is most effective. C. elegans is particularly useful for RNAi research for two reasons: firstly, the effects of gene silencing are generally heritable, and secondly because delivery of the dsRNA is extremely simple. Through a mechanism whose details are poorly understood, bacteria such as E. coli that carry the desired dsRNA can be fed to the worms and will transfer their RNA payload to the worm via the intestinal tract. This ``delivery by feeding'' is just as effective at inducing gene silencing as more costly and time-consuming delivery methods, such as soaking the worms in dsRNA solution and injecting dsRNA into the gonads. Although delivery is more difficult in most other organisms, efforts are also underway to undertake large-scale genomic screening applications in cell culture with mammalian cells.

Approaches to the design of genome-wide RNAi libraries can require more sophistication than the design of a single siRNA for a defined set of experimental conditions. Artificial neural networks are frequently used to design siRNA libraries and to predict their likely efficiency at gene knockdown. Mass genomic screening is widely seen as a promising method for genome annotation and has triggered the development of high-throughput screening methods based on microarrays. However, the utility of these screens and the ability of techniques developed on model organisms to generalize to even closely related species has been questioned, for example from C. elegans to related parasitic nematodes.

Functional genomics using RNAi is a particularly attractive technique for genomic mapping and annotation in plants because many plants are polyploid, which presents substantial challenges for more traditional genetic engineering methods. For example, RNAi has been successfully used for functional genomics studies in bread wheat (which is hexaploid) as well as more common plant model systems Arabidopsis and maize.

\hypertarget{cre-lox-recombination}{%
\section{Cre-Lox recombination}\label{cre-lox-recombination}}

Cre-Lox recombination is a site-specific recombinase technology, used to carry out deletions, insertions, translocations and inversions at specific sites in the DNA of cells. It allows the DNA modification to be targeted to a specific cell type or be triggered by a specific external stimulus. It is implemented both in eukaryotic and prokaryotic systems. The Cre-lox recombination system has been particularly useful to help neuroscientists to study the brain in which complex cell types and neural circuits come together to generate cognition and behaviors. NIH Blueprint for Neuroscience Research has created several hundreds of Cre driver mouse lines which are currently used by the worldwide neuroscience community.

The system consists of a single enzyme, Cre recombinase, that recombines a pair of short target sequences called the Lox sequences. This system can be implemented without inserting any extra supporting proteins or sequences. The Cre enzyme and the original Lox site called the LoxP sequence are derived from bacteriophage P1.

Placing Lox sequences appropriately allows genes to be activated, repressed, or exchanged for other genes. At a DNA level many types of manipulations can be carried out. The activity of the Cre enzyme can be controlled so that it is expressed in a particular cell type or triggered by an external stimulus like a chemical signal or a heat shock. These targeted DNA changes are useful in cell lineage tracing and when mutants are lethal if expressed globally.

The Cre-Lox system is very similar in action and in usage to the FLP-FRT recombination system.

History
Cre-Lox recombination is a special type of site-specific recombination developed by Dr.~Brian Sauer that operated in both mitotic and non-mitotic cells, and was initially used in activating gene expression in mammalian cell lines (DuPont). Subsequently, researchers in the laboratory of Dr.~Jamey Marth demonstrated that Cre-Lox recombination could be used to delete loxP-flanked chromosomal DNA sequences at high efficiency in specific developing T-cells of transgenic animals, with the authors proposing that this approach could be used to define endogenous gene function in specific cell types, indelibly mark progenitors in cell fate determination studies, induce specific chromosomal rearrangements for biological and disease modeling, and determine the roles of early genetic lesions in disease (and phenotype) maintenance.

Shortly thereafter, researchers in the laboratory of Prof.~Klaus Rajewsky reported the production of pluripotent embryonic stem cells bearing a targeted loxP-flanked (floxed) DNA polymerase gene. Combining these advances in collaboration, the laboratories of Drs. Marth and Rajewsky reported in 1994 that Cre-lox recombination could be used for conditional gene targeting. They observed ≈50\% of the DNA polymerase beta gene was deleted in T cells based on DNA blotting. It was unclear whether only one allele in each T-cell or 50\% of T cells had 100\% deletion in both alleles. Researchers have since reported more efficient Cre-Lox conditional gene mutagenesis in the developing T cells by the Marth laboratory in 1995. Incomplete deletion by Cre recombinase is not uncommon in cells when two copies of floxed sequences exist, and allows the formation and study of chimeric tissues. All cell types tested in mice have been shown to undergo transgenic Cre recombination.

Independently, Joe Z. Tsien has pioneered the use of Cre-loxP system for cell type- and region-specific gene manipulation in the adult brain where hundreds of distinct neuron types may exist and nearly all neurons in the adult brain are known to be post-mitotic. Tsien and his colleagues demonstrated Cre-mediated recombination can occur in the post-mitotic pyramidal neurons in the adult mouse forebrain.

These developments have led to a widespread use of conditional mutagenesis in biomedical research, spanning many disciplines in which it becomes a powerful platform for determining gene function in specific cell types and at specific developmental times. In particular, the clear demonstration of its usefulness in precisely defining the complex relationship between specific cells/circuits and behaviors for brain research, has promoted the NIH to initiate the NIH Blueprint for Neuroscience Research Cre-driver mouse projects in early 2000. To date, NIH Blueprint for Neuroscience Research Cre projects have created several hundreds of Cre driver mouse lines which are currently used by the worldwide neuroscience community.

Overview
Cre-Lox recombination involves the targeting of a specific sequence of DNA and splicing it with the help of an enzyme called Cre recombinase. Cre-Lox recombination is commonly used to circumvent embryonic lethality caused by systemic inactivation of many genes. In addition, Cre--Lox recombination provides the best experimental control that presently{[}when?{]} exists in transgenic animal modeling to link genotypes (alterations in genomic DNA) to phenotypes (the biological outcomes).

The Cre-lox system is used as a genetic tool to control site specific recombination events in genomic DNA. This system has allowed researchers to manipulate a variety of genetically modified organisms to control gene expression, delete undesired DNA sequences and modify chromosome architecture.

The Cre protein is a site-specific DNA recombinase that can catalyse the recombination of DNA between specific sites in a DNA molecule. These sites, known as loxP sequences, contain specific binding sites for Cre that surround a directional core sequence where recombination can occur.

When cells that have loxP sites in their genome express Cre, a recombination event can occur between the loxP sites. Cre recombinase proteins bind to the first and last 13 bp regions of a lox site forming a dimer. This dimer then binds to a dimer on another lox site to form a tetramer. Lox sites are directional and the two sites joined by the tetramer are parallel in orientation. The double stranded DNA is cut at both loxP sites by the Cre protein. The strands are then rejoined with DNA ligase in a quick and efficient process. The result of recombination depends on the orientation of the loxP sites. For two lox sites on the same chromosome arm, inverted loxP sites will cause an inversion of the intervening DNA, while a direct repeat of loxP sites will cause a deletion event. If loxP sites are on different chromosomes it is possible for translocation events to be catalysed by Cre induced recombination. Two plasmids can be joined using the variant lox sites 71 and 66.

Cre recombinase
The Cre protein (encoded by the locus originally named as ``Causes recombination'', with ``Cyclization recombinase'' being found in some references) consists of 4 subunits and two domains: The larger carboxyl (C-terminal) domain, and smaller amino (N-terminal) domain. The total protein has 343 amino acids. The C domain is similar in structure to the domain in the Integrase family of enzymes isolated from lambda phage. This is also the catalytic site of the enzyme.

loxP site
loxP (locus of X-over P1) is a site on the bacteriophage P1 consisting of 34 bp. The site includes an asymmetric 8 bp sequence, variable except for the middle two bases, in between two sets of symmetric, 13 bp sequences. The exact sequence is given below; `N' indicates bases which may vary, and lowercase letters indicate bases that have been mutated from the wild-type. The 13 bp sequences are palindromic but the 8 bp spacer is not, thus giving the loxP sequence a certain direction. Usually loxP sites come in pairs for genetic manipulation. If the two loxP sites are in the same orientation, the floxed sequence (sequence flanked by two loxP sites) is excised; however if the two loxP sites are in the opposite orientation, the floxed sequence is inverted. If there exists a floxed donor sequence, the donor sequence can be swapped with the original sequence. This technique is called recombinase-mediated cassette exchange and is a very convenient and time-saving way for genetic manipulation. The caveat, however, is that the recombination reaction can happen backwards, rendering cassette exchange inefficient. In addition, sequence excision can happen in trans instead of a in cis cassette exchange event. The loxP mutants are created to avoid these problems.

Holliday junctions and homologous recombination
During genetic recombination, a Holliday junction is formed between the two strands of DNA and a double-stranded break in a DNA molecule leaves a 3'OH end exposed. This reaction is aided with the endonuclease activity of an enzyme. 5' Phosphate ends are usually the substrates for this reaction, thus extended 3' regions remain. This 3' OH group is highly unstable, and the strand on which it is present must find its complement. Since homologous recombination occurs after DNA replication, two strands of DNA are available, and thus, the 3' OH group must pair with its complement, and it does so, with an intact strand on the other duplex. Now, one point of crossover has occurred, which is what is called a Holliday Intermediate.

The 3'OH end is elongated (that is, bases are added) with the help of DNA Polymerase. The pairing of opposite strands is what constitutes the crossing-over or Recombination event, which is common to all living organisms, since the genetic material on one strand of one duplex has paired with one strand of another duplex, and has been elongated by DNA polymerase. Further cleavage of Holliday Intermediates results in formation of Hybrid DNA.

This further cleavage or `resolvation' is done by a special group of enzymes called Resolvases. RuvC is just one of these Resolvases that have been isolated in bacteria and yeast.

For many years, it was thought that when the Holliday junction intermediate was formed, the branch point of the junction (where the strands cross over) would be located at the first cleavage site. Migration of the branch point to the second cleavage site would then somehow trigger the second half of the pathway. This model provided convenient explanation for the strict requirement for homology between recombining sites, since branch migration would stall at a mismatch and would not allow the second strand exchange to occur. In more recent years, however, this view has been challenged, and most of the current models for Int, Xer, and Flp recombination involve only limited branch migration 1--3 base pairs) of the Holliday intermediate, coupled to an isomerisation event that is responsible for switching the strand cleavage specificity.

Site-specific recombination
Main article: Site-specific recombination
Site-specific recombination (SSR) involves specific sites for the catalyzing action of special enzymes called recombinases. Cre, or cyclic recombinase, is one such enzyme. Site-specific recombination is, thus, the enzyme-mediated cleavage and ligation of two defined deoxynucleotide sequences.

A number of conserved site-specific recombination systems have been described in both prokaryotic and eukaryotic organisms. In general, these systems use one or more proteins and act on unique asymmetric DNA sequences. The products of the recombination event depend on the relative orientation of these asymmetric sequences. Many other proteins apart from the recombinase are involved in regulating the reaction. During site-specific DNA recombination, which brings about genetic rearrangement in processes such as viral integration and excision and chromosomal segregation, these recombinase enzymes recognize specific DNA sequences and catalyse the reciprocal exchange of DNA strands between these sites.

Mechanism of action

Initiation of site-specific recombination begins with the binding of recombination proteins to their respective DNA targets. A separate recombinase recognizes and binds to each of two recombination sites on two different DNA molecules or within the same DNA strand. At the given specific site on the DNA, the hydroxyl group of the tyrosine in the recombinase attacks a phosphate group in the DNA backbone using a direct transesterification mechanism. This reaction links the recombinase protein to the DNA via a phospho-tyrosine linkage. This conserves the energy of the phosphodiester bond, allowing the reaction to be reversed without the involvement of a high-energy cofactor.

Cleavage on the other strand also causes a phospho-tyrosine bond between DNA and the enzyme. At both of the DNA duplexes, the bonding of the phosphate group to tyrosine residues leave a 3' OH group free in the DNA backbone. In fact, the enzyme-DNA complex is an intermediate stage, which is followed by the ligation of the 3' OH group of one DNA strand to the 5' phosphate group of the other DNA strand, which is covalently bonded to the tyrosine residue; that is, the covalent linkage between 5' end and tyrosine residue is broken. This reaction synthesizes the Holliday junction discussed earlier.

In this fashion, opposite DNA strands are joined together. Subsequent cleavage and rejoining cause DNA strands to exchange their segments. Protein-protein interactions drive and direct strand exchange. Energy is not compromised, since the protein-DNA linkage makes up for the loss of the phosphodiester bond, which occurred during cleavage.

Site-specific recombination is also an important process that viruses, such as bacteriophages, adopt to integrate their genetic material into the infected host. The virus, called a prophage in such a state, accomplishes this via integration and excision. The points where the integration and excision reactions occur are called the attachment (att) sites. An attP site on the phage exchanges segments with an attB site on the bacterial DNA. Thus, these are site-specific, occurring only at the respective att sites. The integrase class of enzymes catalyse this particular reaction.

Efficiency of action
Two factors have been shown to affect the efficiency of Cre's excision on the lox pair. First, the nucleotide sequence identity in the spacer region of lox site. Engineered lox variants which differ on the spacer region tend to have varied but generally lower recombination efficiency compared to wildtype loxP, presumably through affecting the formation and resolution of recombination intermediate.

Another factor is the length of DNA between the lox pair. Increasing the length of DNA leads to decreased efficiency of Cre/lox recombination possibly through regulating the dynamics of the reaction. Genetic location of the floxed sequence affects recombination efficiency as well probably by influencing the availability of DNA by Cre recombinase. The choice of Cre driver is also important as low expression of Cre recombinase tends to result in non-parallel recombination. Non-parallel recombination is especially problematic in a fate mapping scenario where one recombination event is designed to manipulate the gene under study and the other recombination event is necessary for activating a reporter gene (usually encoding a fluorescent protein) for cell lineage tracing. Failure to activate both recombination events simultaneously confounds the interpretation of cell fate mapping results.

Temporal control
Another variant that aids in the control of the expression of the gene in terms of its timing is the tamoxifen-inducible system. This is done through the addition of a ligand binding domain(estrogen receptor) on the Cre recombinase which is specifically activated by tamoxifen. Once both sequences of the mutated Cre and the loxP genes are located in one mouse, the absence of tamoxifen will result in the shuttling of the mutated recombinase into the cytoplasm. The protein will stay in this location in its inactivate state until tamoxifen is given. Once tamoxifen is introduced, the mutated recombinase is shuttled back into the nucleus and is able to cleave the lox sites, resulting in a loss of expression. Importantly, sometimes fluorescent reporters can be activated in the absence of tamoxifen, due to leakage of a few Cre recombinase molecules into the nucleus which, in combination with very sensitive reporters, results in unintended cell labelling.

Natural function of the Cre-lox system
The P1 phage is a temperate phage that causes either a lysogenic or lytic cycle when it infects a bacterium. In its lytic state, once its viral genome is injected into the host cell, viral proteins are produced, virions are assembled, and the host cell is lysed to release the phages, continuing the cycle. In the lysogenic cycle the phage genome replicates with the rest of the bacterial genome and is transmitted to daughter cells at each subsequent cell division. It can transition to the lytic cycle by a later event such as UV radiation or starvation.

Phages like the lambda phage use their site specific recombinases to integrate their DNA into the host genome during lysogeny. P1 phage DNA on the other hand, exists as a plasmid in the host. The Cre-lox system serves several functions in the phage: it circularizes the phage DNA into a plasmid, separates interlinked plasmid rings so they are passed to both daughter bacteria equally and may help maintain copy numbers through an alternative means of replication.

The P1 phage DNA when released into the host from the virion is in the form of a linear double stranded DNA molecule. The Cre enzyme targets loxP sites at the ends of this molecule and cyclises the genome. This can also take place in the absence of the Cre lox system with the help of other bacterial and viral proteins. The P1 plasmid is relatively large (≈90Kbp) and hence exists in a low copy number - usually one per cell. If the two daughter plasmids get interlinked one of the daughter cells of the host will lose the plasmid. The Cre-lox recombination system prevents these situations by unlinking the rings of DNA by carrying out two recombination events (linked rings -\textgreater{} single fused ring -\textgreater{} two unlinked rings). It is also proposed that rolling circle replication followed by recombination will allow the plasmid to increase its copy number when certain regulators (repA) are limiting.

Implementation of multiple loxP site pairs
Multiple variants of loxP, in particular lox2272 and loxN, have been used by researchers with the combination of different Cre actions (transient or constitutive) to create a ``Brainbow'' system that allows multi-colouring of mice's brain with four fluorescent proteins.

Another report using two lox variants pair but through regulating the length of DNA in one pair results in stochastic gene activation with regulated level of sparseness.

\hypertarget{crispr-gene-editing}{%
\section{CRISPR gene editing}\label{crispr-gene-editing}}

CRISPR gene editing is a method by which the genomes of living organisms may be edited. It is based on a simplified version of the bacterial CRISPR/Cas (CRISPR-Cas9) antiviral defense system. By delivering the Cas9 nuclease complexed with a synthetic guide RNA (gRNA) into a cell, the cell's genome can be cut at a desired location, allowing existing genes to be removed and/or new ones added.

While genomic editing in eukaryotic cells has been possible using various methods since the 1980s, the methods employed had proved to be inefficient and impractical to implement on a larger scale. Genomic editing leads to irreversible changes to the gene. Working like genetic scissors, the Cas9 nuclease opens both strands of the targeted sequence of DNA to introduce the modification by one of two methods. Knock-in mutations, facilitated via Homology Directed Repair (HDR), is the traditional pathway of targeted genomic editing approaches. This allows for the introduction of targeted DNA damage and repair. HDR employs the use of similar DNA sequences to drive the repair of the break via the incorporation of exogenous DNA to function as the repair template. This method relies on the periodic and isolated occurrence of DNA damage at the target site in order for a repair to commence. Knock-out mutations caused by CRISPR-Cas9 results in the repair of the double-strand break by means of NHEJ (Non-Homologous End Joining). NHEJ can often result in random deletions or insertions at the repair site disrupting or altering gene functionality. Therefore, genomic engineering by CRISPR-Cas9 allows researchers the ability to generate targeted random gene disruption.

Because of this, the precision of genomic editing is a great concern. With the discovery of CRISPR and specifically the Cas9 nuclease molecule, efficient and highly selective editing is now a reality. Cas9 allows for a reliable method of creating a targeted break at a specific location as designated by the crRNA and tracrRna guide strands. Cas9 derived from Streptococcus pyogenes bacteria has facilitated the targeted genomic modification in eukaryotic cells. The ease with which researchers can insert Cas9 and template RNA in order to silence or cause point mutations on specific loci has proved invaluable to the quick and efficient mapping of genomic models and biological processes associated with various genes in a variety of eukaryotes. Newly engineered variants of the Cas9 nuclease have been developed that significantly reduce off-target activity.

CRISPR-Cas9 genome editing techniques have many potential applications, including medicine and crop seed enhancement. The use of CRISPR-Cas9-gRNA complex for genome editing was the AAAS's choice for breakthrough of the year in 2015. Bioethical concerns have been raised about the prospect of using CRISPR for germline editing.

Predecessors
In the early 2000s, researchers developed zinc finger nucleases (ZFNs), synthetic proteins whose DNA-binding domains enable them to create double-stranded breaks in DNA at specific points. In 2010, synthetic nucleases called transcription activator-like effector nucleases (TALENs) provided an easier way to target a double-stranded break to a specific location on the DNA strand. Both zinc finger nucleases and TALENs require the creation of a custom protein for each targeted DNA sequence, which is a more difficult and time-consuming process than that for guide RNAs. CRISPRs are much easier to design because the process requires making only a short RNA sequence.

Whereas RNA interference (RNAi) does not fully suppress gene function, CRISPR, ZFNs and TALENs provide full irreversible gene knockout. CRISPR can also target several DNA sites simultaneously by simply introducing different gRNAs. In addition, CRISPR costs are relatively low.

Genome engineering

CRISPR-Cas9 genome editing is carried out with a Type II CRISPR system. When utilized for genome editing, this system includes Cas9, crRNA, tracrRNA along with an optional section of DNA repair template that is utilized in either non-homologous end joining (NHEJ) or homology directed repair (HDR).

CRISPR-Cas9 often employs a plasmid to transfect the target cells. The main components of this plasmid are displayed in the image and listed in the table. The crRNA needs to be designed for each application as this is the sequence that Cas9 uses to identify and directly bind to the cell's DNA. The crRNA must bind only where editing is desired. The repair template is designed for each application, as it must overlap with the sequences on either side of the cut and code for the insertion sequence.

Multiple crRNAs and the tracrRNA can be packaged together to form a single-guide RNA (sgRNA). This sgRNA can be joined together with the Cas9 gene and made into a plasmid in order to be transfected into cells.

Structure
CRISPR-Cas9 offers a high degree of fidelity and relatively simple construction. It depends on two factors for its specificity: the target sequence and the PAM. The target sequence is 20 bases long as part of each CRISPR locus in the crRNA array. A typical crRNA array has multiple unique target sequences. Cas9 proteins select the correct location on the host's genome by utilizing the sequence to bond with base pairs on the host DNA. The sequence is not part of the Cas9 protein and as a result is customizable and can be independently synthesized.

The PAM sequence on the host genome is recognized by Cas9. Cas9 cannot be easily modified to recognize a different PAM sequence. However this is not too limiting as it is a short sequence and nonspecific (e.g.~the SpCas9 PAM sequence is 5'-NGG-3' and in the human genome occurs roughly every 8 to 12 base pairs).

Once these have been assembled into a plasmid and transfected into cells the Cas9 protein with the help of the crRNA finds the correct sequence in the host cell's DNA and -- depending on the Cas9 variant -- creates a single or double strand break in the DNA.

Properly spaced single strand breaks in the host DNA can trigger homology directed repair, which is less error prone than the non-homologous end joining that typically follows a double strand break. Providing a DNA repair template allows for the insertion of a specific DNA sequence at an exact location within the genome. The repair template should extend 40 to 90 base pairs beyond the Cas9 induced DNA break. The goal is for the cell's HDR process to utilize the provided repair template and thereby incorporate the new sequence into the genome. Once incorporated, this new sequence is now part of the cell's genetic material and passes into its daughter cells.

Many online tools are available to aid in designing effective sgRNA sequences.

Delivery
See also: Transfection
Delivery of Cas9, sgRNA, and associated complexes into cells can occur via viral and non-viral systems. Electroporation of DNA, RNA, or ribonucleocomplexes is a common technique, though it can result in harmful effects on the target cells. Chemical transfection techniques utilizing lipids have also been used to introduce sgRNA in complex with Cas9 into cells. Hard-to-transfect cells (e.g.~stem cells, neurons, and hematopoietic cells) require more efficient delivery systems such as those based on lentivirus (LVs), adenovirus (AdV) and adeno-associated virus (AAV).

Controlled genome editing
Several variants of CRISPR-Cas9 allow gene activation or genome editing with an external trigger such as light or small molecules. These include photoactivatable CRISPR systems developed by fusing light-responsive protein partners with an activator domain and a dCas9 for gene activation, or fusing similar light responsive domains with two constructs of split-Cas9, or by incorporating caged unnatural amino acids into Cas9, or by modifying the guide RNAs with photocleavable complements for genome editing.

Methods to control genome editing with small molecules include an allosteric Cas9, with no detectable background editing, that will activate binding and cleavage upon the addition of 4-hydroxytamoxifen (4-HT), 4-HT responsive intein-linked Cas9s or a Cas9 that is 4-HT responsive when fused to four ERT2 domains. Intein-inducible split-Cas9 allows dimerization of Cas9 fragments and Rapamycin-inducible split-Cas9 system developed by fusing two constructs of split Cas9 with FRB and FKBP fragments. Furthermore, other studies have shown to induce transcription of Cas9 with a small molecule, doxycycline. Small molecules can also be used to improve Homology Directed Repair (HDR), often by inhibiting the Non-Homologous End Joining (NHEJ) pathway. These systems allow conditional control of CRISPR activity for improved precision, efficiency and spatiotemporal control.

Applications
Disease models
Cas9 genomic modification has allowed for the quick and efficient generation of transgenic models within the field of genetics. Cas9 can be easily introduced into the target cells via plasmid transfection along with sgRNA in order to model the spread of diseases and the cell's response and defense to infection. The ability of Cas9 to be introduced in vivo allows for the creation of more accurate models of gene function, mutation effects, all while avoiding the off-target mutations typically observed with older methods of genetic engineering. The CRISPR and Cas9 revolution in genomic modeling doesn't only extend to mammals. Traditional genomic models such as Drosophila melanogaster, one of the first model species, have seen further refinement in their resolution with the use of Cas9. Cas9 uses celtimicrobial therapy and a strategy by which to manipulate bacterial populations. Recent studies suggested a correlation between the interfering of the CRISPR-Cas locus and acquisition of antibiotic resistance This system provides protection of bacteria against invading foreign DNA, such as transposons, bacteriophages and plasmids. This system was shown to be a strong selective pressure for the acquisition of antibiotic resistance and virulence factor in bacterial pathogens.

Therapies based on CRISPR--Cas3 gene editing technology delivered by engineered bacteriophages could be used to destroy targeted DNA in pathogens. Cas3 is more destructive than the better known Cas9

Research suggests that CRISPR is an effective way to limit replication of multiple herpesviruses. It was able to eradicate viral DNA in the case of Epstein-Barr virus (EBV). Anti-herpesvirus CRISPRs have promising applications such as removing cancer-causing EBV from tumor cells, helping rid donated organs for immunocompromised patients of viral invaders, or preventing cold sore outbreaks and recurrent eye infections by blocking HSV-1 reactivation. As of August 2016, these were awaiting testing.

CRISPR may revive the concept of transplanting animal organs into people. Retroviruses present in animal genomes could harm transplant recipients. In 2015, a team eliminated 62 copies of a retrovirus's DNA from the pig genome in a kidney epithelial cell. Researchers recently demonstrated the ability to birth live pig specimens after removing these retroviruses from their genome using CRISPR for the first time.

CRISPR and cancer
As of 2016 CRISPR had been studied in animal models and cancer cell lines, to learn if it can be used to repair or thwart mutated genes that cause cancer.

The first clinical trial involving CRISPR started in 2016. It involved removing immune cells from people with lung cancer, using CRISPR to edit out the gene expressed PD-1, then administrating the altered cells back to the same person. 20 other trials were under way or nearly ready, mostly in China, as of 2017.

In 2016, the United States Food and Drug Administration (FDA) approved a clinical trial in which CRISPR would be used to alter T cells extracted from people with different kinds of cancer and then administer those engineered T cells back to the same people.

Knockdown/activation

Using ``dead'' versions of Cas9 (dCas9) eliminates CRISPR's DNA-cutting ability, while preserving its ability to target desirable sequences. Multiple groups added various regulatory factors to dCas9s, enabling them to turn almost any gene on or off or adjust its level of activity. Like RNAi, CRISPR interference (CRISPRi) turns off genes in a reversible fashion by targeting, but not cutting a site. The targeted site is methylated, epigenetically modifying the gene. This modification inhibits transcription. These precisely placed modifications may then be used to regulate the effects on gene expressions and DNA dynamics after the inhibition of certain genome sequences within DNA. Within the past few years, epigenetic marks in different human cells have been closely researched and certain patterns within the marks have been found to correlate with everything ranging from tumor growth to brain activity. Conversely, CRISPR-mediated activation (CRISPRa) promotes gene transcription. Cas9 is an effective way of targeting and silencing specific genes at the DNA level. In bacteria, the presence of Cas9 alone is enough to block transcription. For mammalian applications, a section of protein is added. Its guide RNA targets regulatory DNA sequences called promoters that immediately precede the target gene.

Cas9 was used to carry synthetic transcription factors that activated specific human genes. The technique achieved a strong effect by targeting multiple CRISPR constructs to slightly different locations on the gene's promoter.

RNA editing
In 2016, researchers demonstrated that CRISPR from an ordinary mouth bacterium could be used to edit RNA. The researchers searched databases containing hundreds of millions of genetic sequences for those that resembled CRISPR genes. They considered the fusobacteria Leptotrichia shahii. It had a group of genes that resembled CRISPR genes, but with important differences. When the researchers equipped other bacteria with these genes, which they called C2c2, they found that the organisms gained a novel defense.

Many viruses encode their genetic information in RNA rather than DNA that they repurpose to make new viruses. HIV and poliovirus are such viruses. Bacteria with C2c2 make molecules that can dismember RNA, destroying the virus. Tailoring these genes opened any RNA molecule to editing.

CRISPR-Cas systems can also be employed for editing of micro-RNA and long-noncoding RNA genes in plants.

Gene drive
Main article: Gene drive
Gene drives may provide a powerful tool to restore balance of ecosystems by eliminating invasive species. Concerns regarding efficacy, unintended consequences in the target species as well as non-target species have been raised particularly in the potential for accidental release from laboratories into the wild. Scientists have proposed several safeguards for ensuring the containment of experimental gene drives including molecular, reproductive, and ecological. Many recommend that immunization and reversal drives be developed in tandem with gene drives in order to overwrite their effects if necessary. There remains consensus that long-term effects must be studied more thoroughly particularly in the potential for ecological disruption that cannot be corrected with reversal drives. As such, DNA computing would be required.

In vitro genetic depletion
Unenriched sequencing libraries often have abundant undesired sequences. Cas9 can specifically deplete the undesired sequences with double strand breakage with up to 99\% efficiency and without significant off-target effects as seen with restriction enzymes. Treatment with Cas9 can deplete abundant rRNA while increasing pathogen sensitivity in RNA-seq libraries.

Prime editing
Prime editing (or base editing) is a CRISPR refinement to accurately insert or delete sections of DNA. The CRISPR edits are not always perfect and the cuts can end up in the wrong place. Both issues are a problem for using the technology in medicine. Prime editing does not cut the double-stranded DNA but instead uses the CRISPR targeting apparatus to shuttle an additional enzyme to a desired sequence, where it converts a single nucleotide into another. The new guide, called a pegRNA, contains an RNA template for a new DNA sequence to be added to the genome at the target location. That requires a second protein, attached to Cas9: a reverse transcriptase enzyme, which can make a new DNA strand from the RNA template and insert it at the nicked site. Those three independent pairing events each provide an opportunity to prevent off-target sequences, which significantly increases targeting flexibility and editing precision. Prime editing was developed by researchers at the Broad Institute of MIT and Harvard in Massachusetts. More work is needed to optimize the methods.

Patents and commercialization
As of November 2013, SAGE Labs (part of Horizon Discovery group) had exclusive rights from one of those companies to produce and sell genetically engineered rats and non-exclusive rights for mouse and rabbit models. By 2015, Thermo Fisher Scientific had licensed intellectual property from ToolGen to develop CRISPR reagent kits.

As of December 2014, patent rights to CRISPR were contested. Several companies formed to develop related drugs and research tools. As companies ramp up financing, doubts as to whether CRISPR can be quickly monetized were raised. In February 2017 the US Patent Office ruled on a patent interference case brought by University of California with respect to patents issued to the Broad Institute, and found that the Broad patents, with claims covering the application of CRISPR-Cas9 in eukaryotic cells, were distinct from the inventions claimed by University of California. Shortly after, University of California filed an appeal of this ruling.

In March 2017, the European Patent Office (EPO) announced its intention to allow broad claims for editing all kinds of cells to Max-Planck Institute in Berlin, University of California, and University of Vienna, and in August 2017, the EPO announced its intention to allow CRISPR claims in a patent application that MilliporeSigma had filed. As of August 2017 the patent situation in Europe was complex, with MilliporeSigma, ToolGen, Vilnius University, and Harvard contending for claims, along with University of California and Broad.

Society and culture
Human germline modification
As of March 2015, multiple groups had announced ongoing research with the intention of laying the foundations for applying CRISPR to human embryos for human germline engineering, including labs in the US, China, and the UK, as well as US biotechnology company OvaScience. Scientists, including a CRISPR co-discoverer, urged a worldwide moratorium on applying CRISPR to the human germline, especially for clinical use. They said ``scientists should avoid even attempting, in lax jurisdictions, germline genome modification for clinical application in humans'' until the full implications ``are discussed among scientific and governmental organizations''. These scientists support further low-level research on CRISPR and do not see CRISPR as developed enough for any clinical use in making heritable changes to humans.

In April 2015, Chinese scientists reported results of an attempt to alter the DNA of non-viable human embryos using CRISPR to correct a mutation that causes beta thalassemia, a lethal heritable disorder. The study had previously been rejected by both Nature and Science in part because of ethical concerns. The experiments resulted in successfully changing only some of the intended genes, and had off-target effects on other genes. The researchers stated that CRISPR is not ready for clinical application in reproductive medicine. In April 2016, Chinese scientists were reported to have made a second unsuccessful attempt to alter the DNA of non-viable human embryos using CRISPR - this time to alter the CCR5 gene to make the embryo HIV resistant.

In December 2015, an International Summit on Human Gene Editing took place in Washington under the guidance of David Baltimore. Members of national scientific academies of America, Britain and China discussed the ethics of germline modification. They agreed to support basic and clinical research under certain legal and ethical guidelines. A specific distinction was made between somatic cells, where the effects of edits are limited to a single individual, versus germline cells, where genome changes could be inherited by descendants. Heritable modifications could have unintended and far-reaching consequences for human evolution, genetically (e.g.~gene/environment interactions) and culturally (e.g.~Social Darwinism). Altering of gametocytes and embryos to generate inheritable changes in humans was defined to be irresponsible. The group agreed to initiate an international forum to address such concerns and harmonize regulations across countries.

In November 2018, Jiankui He announced that he had edited two human embryos, to attempt to disable the gene for CCR5, which codes for a receptor that HIV uses to enter cells. He said that twin girls, Lulu and Nana, had been born a few weeks earlier. He said that the girls still carried functional copies of CCR5 along with disabled CCR5 (mosaicism) and were still vulnerable to HIV. The work was widely condemned as unethical, dangerous, and premature. An international group of scientists called for a global moratorium on genetically editing human embryos.

Policy barriers to genetic engineering
Policy regulations for the CRISPR-Cas9 system vary around the globe. In February 2016, British scientists were given permission by regulators to genetically modify human embryos by using CRISPR-Cas9 and related techniques. However, researchers were forbidden from implanting the embryos and the embryos were to be destroyed after seven days.

The US has an elaborate, interdepartmental regulatory system to evaluate new genetically modified foods and crops. For example, the Agriculture Risk Protection Act of 2000 gives the USDA the authority to oversee the detection, control, eradication, suppression, prevention, or retardation of the spread of plant pests or noxious weeds to protect the agriculture, environment and economy of the US. The act regulates any genetically modified organism that utilizes the genome of a predefined ``plant pest'' or any plant not previously categorized. In 2015, Yinong Yang successfully deactivated 16 specific genes in the white button mushroom, to make them non-browning. Since he had not added any foreign-species (transgenic) DNA to his organism, the mushroom could not be regulated by the USDA under Section 340.2. Yang's white button mushroom was the first organism genetically modified with the CRISPR-Cas9 protein system to pass US regulation. In 2016, the USDA sponsored a committee to consider future regulatory policy for upcoming genetic modification techniques. With the help of the US National Academies of Sciences, Engineering and Medicine, special interests groups met on April 15 to contemplate the possible advancements in genetic engineering within the next five years and any new regulations that might be needed as a result. The FDA in 2017 proposed a rule that would classify genetic engineering modifications to animals as ``animal drugs'', subjecting them to strict regulation if offered for sale, and reducing the ability for individuals and small businesses to make them profitably.

In China, where social conditions sharply contrast with the west, genetic diseases carry a heavy stigma. This leaves China with fewer policy barriers to the use of this technology.

Recognition
In 2012, and 2013, CRISPR was a runner-up in Science Magazine's Breakthrough of the Year award. In 2015, it was the winner of that award. CRISPR was named as one of MIT Technology Review's 10 breakthrough technologies in 2014 and 2016. In 2016, Jennifer Doudna, Emmanuelle Charpentier, along with Rudolph Barrangou, Philippe Horvath, and Feng Zhang won the Gairdner International award. In 2017, Jennifer Doudna and Emmanuelle Charpentier were awarded the Japan Prize for their revolutionary invention of CRISPR-Cas9 in Tokyo, Japan. In 2016, Emmanuelle Charpentier, Jennifer Doudna, and Feng Zhang won the Tang Prize in Biopharmaceutical Science.


